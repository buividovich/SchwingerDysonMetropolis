\documentclass[12pt]{article}

\usepackage{amsmath}
\usepackage{amsfonts}

\newcommand{\comment}[1]{}

\newcommand{\lr}[1]{ \left( #1 \right) }
\newcommand{\lrs}[1]{ \left[ #1 \right] }
\newcommand{\lrc}[1]{ \left\{ #1 \right\} }
\newcommand{\vev}[1]{ \langle \, #1 \, \rangle }

\newcommand{\Tr}{ {\rm Tr} \, }
\newcommand{\tr}{ {\rm Tr} \, }
\newcommand{\re}{ {\rm Re} \, }
\newcommand{\im}{ {\rm Im} \, }
\renewcommand{\Re}{ {\rm Re} \, }
\renewcommand{\Im}{ {\rm Im} \, }

\newcommand{\rvac}{ \, | 0 \rangle }
\newcommand{\lvac}{ \langle 0 | \, }
\newcommand{\ket}[1]{ \, | #1 \rangle }
\newcommand{\bra}[1]{ \langle #1 | \, }

\newcommand{\diag}[1]{ {\rm diag} \, \left( #1 \right) }
\newcommand{\const}{ {\rm const}}
\renewcommand{\det}[1]{ {\rm det} \left( #1 \right) }

\newcommand{\sign}{ {\rm sign} \,  }
\newcommand{\sh}{ {\rm sh} \,  }
\newcommand{\ch}{ {\rm ch} \,  }
\renewcommand{\th}{ {\rm th} \,  }
\newcommand{\hodge}{{}^{*}}
\newcommand{\expa}[1]{ \exp{\left( #1 \right)} }
\newcommand{\abs}[1]{| #1 |}

\begin{document}

\section{Action in terms of stereographic variables}
\label{sec:action_stereo}

 We parameterize the $U\lr{N}$ the field variables as
\begin{eqnarray}
\label{stereographic_projection}
 g_x
 =
 \frac{1 + i \alpha \phi_x}{1 - i \alpha \phi_x}
 =
 1 + 2 \sum\limits_{k=1}^{+\infty} \lr{i \alpha}^k \phi_x^k
,
\end{eqnarray}
where $\phi_x$ are the Hermitian matrices. The action of the PCM is now rewritten as
\begin{eqnarray}
\label{pcm_action0}
 S_0\lrs{g_x} = \lambda^{-1} \sum\limits_{x,y} D_{x y} \tr\lr{g^{\dag}_x g_y}
 = \nonumber\\ =
 4 \lambda^{-1} \sum\limits_{k,l=1}^{+\infty} \lr{-i \alpha}^k \lr{i \alpha}^l
      \sum\limits_{x,y} D_{xy} \tr\lr{\phi_x^k \, \phi_y^l}
 = \nonumber\\ =
 4 \lambda^{-1} \sum\limits_{\substack{k,l=1\\k+l=2 n}}^{+\infty}
  \lr{-1}^{\frac{k-l}{2}} \alpha^{k+l}
       \sum\limits_{x,y} D_{xy} \tr\lr{\phi_x^k \, \phi_y^l} .
\end{eqnarray}
In addition to the above ``bare'' action, we also have to consider the Jacobian of the transformation $g_x \rightarrow \phi_x$:
\begin{eqnarray}
\label{stereo_jacobian}
 \int\limits_{U\lr{N}} dg_x = \int\limits_{\mathbb{H}^{N\times N}} d\phi_x \expa{-N \tr\ln\lr{1 + \alpha^2 \phi_x^2}} .
\end{eqnarray}
We now see that in order to get the canonical form of the free (quadratic) part of the action, we have to use $\alpha^2 = \frac{\lambda}{8}$. Taking (\ref{pcm_action0}) and (\ref{stereo_jacobian}) together, we obtain the following representation of the action:
\begin{eqnarray}
\label{pcm_action}
 S\lrs{\phi_x} =
 \frac{1}{2} \sum\limits_{x,y} \lr{D_{xy} + \frac{\lambda}{4} \delta_{x y}} \tr\lr{\phi_x \phi_y}
 + \nonumber \\ +
 \sum\limits_{k=2}^{+\infty} \lr{-1}^{k-1} \frac{\lambda^{k}}{k \, 8^k} \sum\limits_x \tr \phi_x^{2 k}
 + \nonumber \\ +
 \sum\limits_{\substack{k,l=1\\k+l=2 n,k+l>2}}^{+\infty}
  \lr{-1}^{\frac{k-l}{2}} \frac{4 \, \lambda^{\frac{k+l-2}{2}}}{8^{\frac{k+l}{2}}}
       \sum\limits_{x,y} D_{xy} \tr\lr{\phi_x^k \, \phi_y^l}
\end{eqnarray}
It is instructive to explicitly write out the terms involving interaction vertices with powers of $\phi$ equal to four:
\begin{eqnarray}
\label{interaction_action_phi4}
 S_I^{\lr{4}} = - \frac{\lambda^2}{128} \sum\limits_x \tr \phi_x^4
 + \nonumber \\ +
 \frac{\lambda}{16} \sum\limits_{x, y} D_{x y} \lr{\tr\lr{\phi_x^2 \phi_y^2} - D_{x y} \tr\lr{\phi_x^3 \phi_y} - D_{x y} \tr\lr{\phi_x \phi_y^3} }
\end{eqnarray}

\section{Schwinger-Dyson equations in coordinate space}
\label{sec:sd_coordinate}

 In order to derive the SD equations, we consider the following full derivative:
\begin{eqnarray}
\label{sd_eqs0}
 \delta_{ik} \delta_{jl} \, \int \mathcal{D}\phi_x \frac{\partial}{\partial \phi_{x_1 \, ij}}
 \lr{
 \lrs{\phi_{x_2} \ldots \phi_{x_n}}_{kl} e^{-S\lrs{\phi_x}}
 } = 0
\end{eqnarray}
From this we obtain for correlators with $n = 2$
\begin{eqnarray}
\label{sd_eqs_G2_coord}
 \delta_{x_1 x_2} - \vev{\phi_{x_2} \frac{\partial S}{\partial \phi_{x_1}} } = 0 .
\end{eqnarray}
and for higher-order correlators with $n > 2$ (in fact, $n \geq 4$)
\begin{eqnarray}
\label{sd_eqs_Gn_coord}
 \delta_{x_1 x_2} \vev{\phi_{x_3} \ldots \phi_{x_n}}
 +
 \delta_{x_1 x_n} \vev{\phi_{x_2} \ldots \phi_{x_{n-1}}}
 + \nonumber \\ +
 \sum\limits_{A=3}^{n-1} \delta_{x_1 x_A} \,
 \vev{\phi_{x_2} \ldots \phi_{x_{A-1}}}
 \vev{\phi_{x_{A+1}} \ldots \phi_{x_n}}
 -
 \vev{\phi_{x_2} \ldots \phi_{x_n} \frac{\partial S}{\partial \phi_{x_1}} } = 0 .
\end{eqnarray}
The derivative of the action can be represented as follows:
\begin{eqnarray}
\label{dSdphi_coord}
 \frac{\partial S\lrs{\phi}}{\partial \phi_x}
 =
 \sum\limits_{y} \lr{D_{xy} + \frac{\lambda}{4} \delta_{x y}} \phi_y
 +
 \sum\limits_{l=1}^{+\infty} \lr{-1}^l \frac{2 \lambda^{l+1}}{8^{l+1}} \phi_x^{2 l + 1}
 + \nonumber \\ +
 \sum\limits_{\substack{k,l=1\\k+l=2 n,k+l>2}}^{+\infty}
 \lr{-1}^{\frac{k-l}{2}} \lr{\frac{\lambda}{8}}^{\frac{k+l-2}{2}}
 \sum\limits_y D_{x y}
 \sum\limits_{m=0}^{k-1}
 \phi_x^m \phi_y^l \phi_x^{k-m-1}
 = \nonumber \\ =
 \sum\limits_y \lrs{G_0^{-1}}_{x y} \, \phi_y
 + \nonumber \\ +
 \sum\limits_{k=1}^{+\infty} \lr{-\alpha^2}^k
 \lr{
  \frac{\lambda}{4} \phi_x^{2 k + 1}
  +
  \sum\limits_{l=1}^{2 k + 1} \lr{-1}^{l-1}
  \sum\limits_{m=0}^{2 k + 1 - l}
  \sum\limits_y D_{x y}
  \phi_x^m \phi_y^l \phi_x^{2 k + 1 - l - m}
 }
\end{eqnarray}

\subsection{Schwinger-Dyson equations in algorithmic form}
\label{subsec:sd_coord_alg}

We now assume the expansion 
\begin{eqnarray}
\label{expansion_assumed}
 \vev{x_0 \ldots x_{2 n - 1}} = \sum\limits_{m=0}^{+\infty}
 \vev{x_0 \ldots x_{2 n - 1}}_m \lr{-\alpha^2}^m 
\end{eqnarray}

\begin{eqnarray}
\label{sd_g2_coord_algorithmic}
 \vev{x_0 x_1}_m = G_0\lr{x_0, x_1} \delta_{m, 0}
 - \nonumber \\ -
 \sum\limits_{k=1}^{m}
 G_0\lr{x_0, z}
 \left(
 \frac{\lambda}{4} \vev{z^{2 k + 1} x_2}_{m - k}
 + \right. \nonumber \\ \left. 
 \sum\limits_{l=1}^{2 k + 1} \lr{-1}^{l-1}
 \sum\limits_{m=0}^{2 k + 1 - l}
 \sum\limits_y D_{z y}
 \vev{z^m y^l z^{2k+1-l-m} x_1}_{m - k}
 \right)
\end{eqnarray}

\begin{eqnarray}
\label{sd_gn_coord_algorithmic}
 \vev{x_0 x_1 \ldots x_{2 n - 1}}_m 
 = \nonumber \\ =
 G_0\lr{x_0, x_1} \vev{x_2 x_3 \ldots x_{2 n - 1}}_m
 +
 G_0\lr{x_0, x_{2 n - 1}} \vev{x_1 x_3 \ldots x_{2 n - 2}}_m
 + \nonumber \\ +
 \sum\limits_{A=3}^{2 n - 3}
 G_0\lr{x_0, x_A}
 \sum\limits_{m'=0}^m
 \vev{x_1 x_2 \ldots x_{A-1}}_m'
 \vev{x_{A+1} \ldots x_{2 n - 1}}_{m - m'}
 - \nonumber \\ -
 \sum\limits_{k=1}^{m} \sum\limits_z
 G_0\lr{x_0, z}
 \left(
 \frac{\lambda}{4} \vev{z^{2 k + 1} x_1 \ldots x_{2 n - 1}}_{m - k}
 + \right. \nonumber \\ \left.
 \sum\limits_{l=1}^{2 k + 1} \lr{-1}^{l-1}
 \sum\limits_{m=0}^{2 k + 1 - l}
 \sum\limits_y D_{z y}
 \vev{z^m y^l z^{2k+1-l-m} x_1 \ldots x_{2 n - 1}}_{m - k}
 \right)
\end{eqnarray}


\subsection{Correlators of $g_x$ variables}
\label{subsec:gx_correlators_coordinate}

\begin{eqnarray}
\label{gx_vev_coordinate}
 \vev{g_x} = 1 + 2 \sum\limits_{n=1}^{+\infty} \lr{- \, \alpha^2}^n \vev{\phi_x^{2 n}}
\end{eqnarray}

\begin{eqnarray}
\label{gx_vev_coordinate}
 \vev{g^{\dag}_0 g_x}
 =
 -1 + 2 \vev{g_x}
 +
 4 \sum\limits_{n=1}^{+\infty} \lr{-\alpha^2}^n
 \sum\limits_{k=1}^{2 n - 1} \lr{-1}^k
 \vev{\phi_x^k \, \phi_0^{2n - k}}
\end{eqnarray}


\subsection{Observables in coordinate space}

\section{Schwinger-Dyson equations in momentum space}
\label{sec:sd_momentum}

 Let us now define the momentum-space field operators and correlators as
\begin{eqnarray}
\label{momentum_space_fields}
 \phi_p = \frac{1}{V} \sum\limits_x e^{i p x} \phi_x,
 \quad
 \phi_x = \sum\limits_p e^{-i p x} \phi_p ,
 \nonumber \\
 \vev{p_1 \ldots p_n} \equiv \vev{\phi_{p_1} \ldots \phi_{p_n}} .
\end{eqnarray}
In terms of the momentum-space fields $\phi_p$, the SD equations (\ref{sd_eqs_G2_coord}) and (\ref{sd_eqs_Gn_coord}) read
\begin{eqnarray}
\label{sd_eqs_G2_mom}
 \vev{\frac{\partial S}{\partial \phi_{p_1}} \phi_{p_2} } = \frac{\delta_{p_1 + p_2}}{V}  .
\end{eqnarray}
and for higher-order correlators with $n > 2$ (in fact, $n \geq 4$)
\begin{eqnarray}
\label{sd_eqs_Gn_mom}
 \vev{\frac{\partial S}{\partial \phi_{p_1}} \phi_{p_2} \ldots \phi_{p_n} }
 =
 \frac{\delta_{p_1 + p_2}}{V} \vev{\phi_{p_3} \ldots \phi_{p_n}}
 +
 \frac{\delta_{p_1 + p_2}}{V} \vev{\phi_{p_2} \ldots \phi_{p_{n-1}}}
 + \nonumber \\ +
 \sum\limits_{A=3}^{n-1} \frac{\delta_{p_1 + p_A}}{V} \,
 \vev{\phi_{p_2} \ldots \phi_{p_{A-1}}}
 \vev{\phi_{p_{A+1}} \ldots \phi_{p_n}} .
\end{eqnarray}
The Fourier transform of the derivative of the action $\frac{\partial S}{\partial \phi_p}$ reads:
\begin{eqnarray}
\label{dSdphi_mom}
 \frac{\partial S\lrs{\phi}}{\partial \phi_p}
 \equiv
 \frac{1}{V} \sum\limits_p e^{i p x} \frac{\partial S\lrs{\phi}}{\partial \phi_x}
 =
 \lr{D\lr{p} + \frac{\lambda}{4}} \phi_p
 - \nonumber \\ -
 \sum\limits_{l=1}^{+\infty} \lr{-1}^{l-1} \frac{2 \lambda^{l+1}}{8^{l+1}}
 \sum\limits_{q_1 \ldots q_{2 l + 1}}
 \delta\lr{p - q_1 - \ldots - q_{2 l + 1}}
 \phi_{q_1} \ldots \phi_{q_{2 l + 1}}
 - \nonumber \\ -
 \sum\limits_{\substack{k,l=1\\k+l=2 n,k+l>2}}^{+\infty}
 \lr{-1}^{\frac{k-l}{2}-1} \lr{\frac{\lambda}{8}}^{\frac{k+l-2}{2}}
 \sum\limits_{m=0}^{k-1}
 \nonumber \\
 \sum\limits_{q_1, \ldots, q_{k-1}}
 \sum\limits_{k_1, \ldots, k_l}
 \delta\lr{p - \bar{q} - \bar{k}} \, D\lr{\bar{k}} \,
 \phi_{q_1} \ldots \phi_{q_m}
 \phi_{k_1} \ldots \phi_{k_l}
 \phi_{q_{m+1}} \ldots \phi_{q_{k-1}} ,
\end{eqnarray}
where we have denoted $\bar{k} = k_1 + \ldots + k_l$ and similarly for $q$.
It is also instructive to write out explicitly the interaction terms with exactly three $\phi$ variables (which correspond to the simplest nontrivial interaction vertices in the diagrammatic rules of the theory):
\begin{eqnarray}
\label{dsdPhi_phi4}
 \frac{\partial S_I^{\lr{4}}\lrs{\phi}}{\partial \phi_p}
 =
 - \frac{\lambda}{8}
 \sum\limits_{q_1, q_2, q_3}
 \delta\lr{p - q_1 - q_2 - q_3} \,
 V\lr{q_1, q_2, q_3} \,
 \phi_{q_1} \phi_{q_2} \phi_{q_3} ,
\end{eqnarray}
where we have defined the ``vertex function'' for the four-momentum vertex
\begin{eqnarray}
\label{vertex_function4}
 V\lr{q_1, q_2, q_3} =
 \left(\frac{\lambda}{4} + D\lr{q_1 + q_2 + q_3} - D\lr{q_1 + q_2}
 - \right. \nonumber \\ \left. -
 D\lr{q_2 + q_3} + D\lr{q_1} + D\lr{q_2} + D\lr{q_3}\right)
\end{eqnarray}
Interestingly, at small momenta, when all the $D\lr{q}$ functions can be approximated by $q^2$, the vertex function reduces to $\frac{\lambda}{4} + \lr{q_1 + q_3}^2$ - which is manifestly positive and \textbf{very good for the sign problem in MC}!!!

\subsection{First-order correction to the renormalized mass}

 Plugging the equation (\ref{dsdPhi_phi4}) into the lowest-order SD equation (\ref{sd_eqs_G2_mom}), we get for the first order nontrivial correction to the Green's function:
\begin{eqnarray}
\label{g1}
 g_1\lr{p} = g_0\lr{p} - g_0^2\lr{p} \Sigma_1\lr{p} ,
\end{eqnarray}
where we have introduced the first-order correction to the self-energy:
\begin{eqnarray}
\label{self_energy}
 \Sigma_1\lr{p} = - \frac{\lambda}{8} \frac{1}{V} \sum\limits_q g_0\lr{q} \lr{V\lr{-q, q, -p} + V\lr{-p, q, -q}}
 = \nonumber \\ =
 - 2 \frac{\lambda}{8} \int \frac{d^2 q}{\lr{2 \pi}^2} g_0\lr{q} \lr{\frac{\lambda}{4} + 2 D\lr{p} + 2 D\lr{q} - D\lr{q - p}}
\end{eqnarray}
Interestingly, we note that $\Sigma^1\lr{0} = -\frac{\lambda}{4}$, thus it exactly cancels the bare mass term!

For the expectation values of the $g_x$ operators, to the lowest orders in $\lambda$ we have
\begin{eqnarray}
\label{gx_vev_lowest}
 \vev{g_x} = 1 - \frac{\lambda}{4} \vev{\phi_x^2} + \frac{\lambda^2}{32} \vev{\phi_x^4}
\end{eqnarray}

\begin{eqnarray}
\label{gxgy_vev_lowest}
 \vev{g^{\dag}_x g_y} =
 -1 + 2 \vev{g_x} + \frac{\lambda}{2} \vev{\phi_x \phi_y}
 + \nonumber \\ +
   \frac{\lambda^2}{16} \vev{\phi_x^2 \phi_y^2}
 - \frac{\lambda^2}{16} \lr{\vev{\phi_x \phi_y^3} + \vev{\phi_x^3 \phi_y}}
\end{eqnarray}


\subsection{Vertex functions}
\label{subsec:vertex_func}

 The equation (\ref{dSdphi_mom}) can be also compactly represented in terms of the vertex functions:
\begin{eqnarray}
\label{dSdphi_mom_vertex}
 \frac{\partial S\lrs{\phi}}{\partial \phi_p}
 =
 \lr{D\lr{p} + \frac{\lambda}{4}} \phi_p
 + \nonumber \\ +
 \sum\limits_{n=1}^{+\infty} \lr{-\frac{\lambda}{8}}^{n}
 \sum\limits_{q_1 \ldots q_{2 n + 1}} \delta\lr{p - q_1 - \ldots - q_{2 n + 1}}
 V\lr{q_1, \ldots, q_{2 n + 1}} ,
\end{eqnarray}
where the arbitrary-order vertex functions are defined as
\begin{eqnarray}
\label{vertex_func_def}
 V\lr{q_1, \ldots, q_{2 n + 1}}
 =
 \frac{\lambda}{4} +
 \sum\limits_{l=1}^{2 n + 1} \lr{-1}^{l-1}
 \sum\limits_{m=0}^{2 n + 1 - l}
 D\lr{q_{m+1} + \ldots + q_{m+l}}
\end{eqnarray}
For algorithmic purposes, it is more convenient to use the following representation:
\begin{eqnarray}
\label{vertex_func_def_algorithmic}
 V\lr{q_0, \ldots, q_{2 n}}
 =
 \frac{\lambda}{4} +
 \sum\limits_{m=0}^{2 n}
 \sum\limits_{l=0}^{2 n - m}
 \lr{-1}^{l}
 D\lr{q_{m} + \ldots + q_{m+l}}
\end{eqnarray}

Correspondingly, the SD equations can be written in the following compact form:
\begin{eqnarray}
\label{sd_eqs_G2_mom_vertex}
 \vev{p_1 p_2} = \frac{\delta\lr{p_1 + p_2}}{V} \, G_0\lr{p_1}
 - \nonumber \\ -
 G_0\lr{p_1} \sum\limits_{m=1}^{+\infty}
 \lr{-\frac{\lambda}{8}}^m
 \times \nonumber \\ \times
 \sum\limits_{q_1, \ldots, q_{2 m + 1}}
 \delta\lr{p_1 - Q}
 V\lr{q_1, \ldots, q_{2 m + 1}}
 \vev{q_1 \ldots q_{2 m + 1} \, p_2} .
\end{eqnarray}
and for higher-order correlators with $n > 2$ (in fact, $n \geq 4$)
\begin{eqnarray}
\label{sd_eqs_Gn_mom_vertex}
 \vev{p_1 p_2 \ldots p_n}
 =
 \frac{\delta\lr{p_1 + p_2}}{V} \, G_0\lr{p_1} \, \vev{p_3 \ldots p_n}
 + \nonumber \\ +
 \frac{\delta\lr{p_1 + p_n}}{V} \, G_0\lr{p_1} \, \vev{p_2 \ldots p_{n-1}}
 + \nonumber \\ +
 \sum\limits_{A=4}^{n-2}
 \frac{\delta\lr{p_1 + p_A}}{V} \, G_0\lr{p_1} \,
 \vev{p_2 \ldots p_{A-1}} \, \vev{p_{A+1} \ldots p_n}
 - \nonumber \\ -
 G_0\lr{p_1} \sum\limits_{m=1}^{+\infty}
 \lr{-\frac{\lambda}{8}}^m
 \times \nonumber \\ \times
 \sum\limits_{q_1, \ldots, q_{2 m + 1}}
 \delta\lr{p_1 - Q}
 V\lr{q_1, \ldots, q_{2 m + 1}}
 \vev{q_1 \ldots q_{2 m + 1} \, p_2  \ldots p_n} ,
\end{eqnarray}
where we have denoted $Q = q_1 + \ldots + q_{2 n + 1}$.

\subsection{SD equations in ``Algorithmic'' form}
\label{subsec:sds_algorithmic}

 For the purpose of algorithm development, let us define the formal series in powers of $\alpha$ as:
\begin{eqnarray}
\label{alpha_series_def}
 \vev{p_0 \ldots p_{2 n - 1}}
 =
 \sum\limits_{m=0}^{+\infty}
 \vev{p_0 \ldots p_{2 n - 1}}_m \alpha^m
\end{eqnarray}

The SD equations (\ref{sd_eqs_G2_mom_vertex}) and (\ref{sd_eqs_Gn_mom_vertex}) can be now rewritten as the recursive relations between successive coefficients in the series (\ref{alpha_series_def}):
\begin{eqnarray}
\label{sd_eqs_G2_mom_vertex_alg}
 \vev{p_0 p_1}_m = \frac{\delta\lr{p_0 + p_1}}{V} \, G_0\lr{p_0} \, \delta_{m,0}
 + \nonumber \\ +
 \lr{1 - \delta_{m,0}} \,
 G_0\lr{p_0} \sum\limits_{k=1}^{m} \lr{-1}^{k-1}
 \times \nonumber \\ \times
 \sum\limits_{q_0, \ldots, q_{2 k}}
 \delta\lr{p_0 - Q}
 V\lr{q_0, \ldots, q_{2 k}}
 \vev{q_0 \ldots q_{2 k} \, p_1}_{m-k} .
\end{eqnarray}
and for higher-order correlators with $n > 1$
\begin{eqnarray}
\label{sd_eqs_Gn_mom_vertex_alg}
 \vev{p_0 p_1 \ldots p_{2n-1}}_m
 =
 \frac{\delta\lr{p_0 + p_1}}{V} \, G_0\lr{p_0} \, \vev{p_2 \ldots p_{2n-1}}_m
 + \nonumber \\ +
 \frac{\delta\lr{p_0 + p_{2n-1}}}{V} \, G_0\lr{p_0} \, \vev{p_1 \ldots p_{2n-2}}_m
 + \nonumber \\ +
 \sum\limits_{A=1}^{n-2}
 \frac{\delta\lr{p_0 + p_{2A+1}}}{V} \, G_0\lr{p_0} \,
 \sum\limits_{k=0}^m
 \vev{p_1 \ldots p_{2A}}_k \, \vev{p_{2 A+2} \ldots p_{2n-1}}_{m-k}
 + \nonumber \\ +
 \lr{1 - \delta_{m,0}} \,
 G_0\lr{p_0} \sum\limits_{k=1}^{m} \lr{-1}^{k-1}
 \times \nonumber \\ \times
 \sum\limits_{q_0, \ldots, q_{2 k}}
 \delta\lr{p_0 - Q}
 V\lr{q_0, \ldots, q_{2 k}}
 \vev{q_0 \ldots q_{2 k} \, p_1  \ldots p_{2n-1}}_{m-k} ,
\end{eqnarray}
where we have denoted $Q = q_0 + \ldots + q_{2 k}$.


\subsection{Correlators of $g_x$ variables}
\label{subsec:gx_correlators_momentum}

 An important criterion of whether the $SU\lr{N} \times SU\lr{N}$ symmetry of the action is restored dynamically is the expectation value of a single field variable $\vev{g_x}$, which should tend to zero as the approximation becomes more and more exact. Expressing $\vev{g_x}$ in terms of the variables $\phi_x$, we get:
\begin{eqnarray}
\label{gx_vev_momentum}
 \vev{g_x} \equiv \frac{1}{V} \sum\limits_x \vev{g_x}
 =
 1 + 2 \sum\limits_{k=1}^{+\infty} \lr{-\frac{\lambda}{8}}^k \frac{1}{V} \sum\limits_x \vev{\phi_x^{2 k}}
 = \nonumber \\ =
 1 + 2 \sum\limits_{k=1}^{+\infty} \lr{-\frac{\lambda}{8}}^k
 \sum\limits_{q_1 \ldots q_{2 k}}
 \delta\lr{q_1 + \ldots + q_{2 k}}
 \vev{\phi_{q_1} \ldots \phi_{q_{2k}}} .
\end{eqnarray}
In the same way, we can express the correlator $\vev{g^{\dag}_x g_y}$ in coordinate space as:
\begin{eqnarray}
\label{gx_vev_space}
 \vev{g^{\dag}_x g_y}
 =
 2 \vev{g_x} - 1
 +
 4 \sum\limits_{k,l=1}^{+\infty}
 \lr{-1}^{\frac{k-l}{2}} \, \lr{\frac{\lambda}{8}}^{\frac{k+l}{2}}
 \vev{\phi_x^k \phi_y^l} .
\end{eqnarray}
Performing the Fourier transform and assuming translational invariance, we get
\begin{eqnarray}
\label{gx_vev_momentum}
 \vev{g^{\dag}_x g_0}
 =
 2 \vev{g_x} - 1
 + \nonumber \\ +
 4 \sum\limits_{n=1}^{+\infty} \lr{-\lambda/8}^n
 \sum\limits_{q_1 \ldots q_{2 n}}
 \Gamma\lr{x; \, q_1, \ldots, q_{2 n}}
 \vev{\phi_{q_1} \ldots \phi_{q_{2 n}}} ,
 \nonumber \\
 \Gamma\lr{x; \, q_1, \ldots, q_{2 n}} =
 \sum\limits_{k=1}^{2 n - 1} \lr{-1}^k \cos\lr{\lr{q_1 + \ldots + q_k} x}
\end{eqnarray}

\end{document}
