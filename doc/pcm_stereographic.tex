\documentclass[12pt]{article}

\usepackage{amsmath}
\usepackage{amsfonts}
\usepackage{color}
\usepackage{lscape}

\newcommand{\comment}[1]{}

\newcommand{\lr}[1]{ \left( #1 \right) }
\newcommand{\lrs}[1]{ \left[ #1 \right] }
\newcommand{\lrc}[1]{ \left\{ #1 \right\} }
\newcommand{\vev}[1]{ \langle \, #1 \, \rangle }

\newcommand{\Tr}{ {\rm Tr} \, }
\newcommand{\tr}{ {\rm Tr} \, }
\newcommand{\re}{ {\rm Re} \, }
\newcommand{\im}{ {\rm Im} \, }
\renewcommand{\Re}{ {\rm Re} \, }
\renewcommand{\Im}{ {\rm Im} \, }

\newcommand{\rvac}{ \, | 0 \rangle }
\newcommand{\lvac}{ \langle 0 | \, }
\newcommand{\ket}[1]{ \, | #1 \rangle }
\newcommand{\bra}[1]{ \langle #1 | \, }

\newcommand{\diag}[1]{ {\rm diag} \, \left( #1 \right) }
\newcommand{\const}{ {\rm const}}
\renewcommand{\det}[1]{ {\rm det} \left( #1 \right) }

\newcommand{\sign}{ {\rm sign} \,  }
\newcommand{\sh}{ {\rm sh} \,  }
\newcommand{\ch}{ {\rm ch} \,  }
\renewcommand{\th}{ {\rm th} \,  }
\newcommand{\hodge}{{}^{*}}
\newcommand{\expa}[1]{ \exp{\left( #1 \right)} }
\newcommand{\abs}[1]{| #1 |}

\newcommand{\red}[1]{ \textcolor{red}{#1} }

\begin{document}

\section{Action in terms of stereographic variables}
\label{sec:action_stereo}

%TODO: \cite{Kurkela:16:1}

 We parameterize the $U\lr{N}$ the field variables as
\begin{eqnarray}
\label{stereographic_projection}
 g_x
 =
 \frac{1 + i \alpha \phi_x}{1 - i \alpha \phi_x}
 =
 1 + 2 \sum\limits_{k=1}^{+\infty} \lr{i \alpha}^k \phi_x^k
,
\end{eqnarray}
where $\phi_x$ are the Hermitian matrices. The action of the PCM is now rewritten as
\begin{eqnarray}
\label{pcm_action0}
 S_0\lrs{g_x} = \lambda^{-1} \sum\limits_{x,y} D_{x y} \tr\lr{g^{\dag}_x g_y}
 = \nonumber\\ =
 4 \lambda^{-1} \sum\limits_{k,l=1}^{+\infty} \lr{-i \alpha}^k \lr{i \alpha}^l
      \sum\limits_{x,y} D_{xy} \tr\lr{\phi_x^k \, \phi_y^l}
 = \nonumber\\ =
 4 \lambda^{-1} \sum\limits_{\substack{k,l=1\\k+l=2 n}}^{+\infty}
  \lr{-1}^{\frac{k-l}{2}} \alpha^{k+l}
       \sum\limits_{x,y} D_{xy} \tr\lr{\phi_x^k \, \phi_y^l} .
\end{eqnarray}
In addition to the above ``bare'' action, we also have to consider the Jacobian of the transformation $g_x \rightarrow \phi_x$:
\begin{eqnarray}
\label{stereo_jacobian}
 \int\limits_{U\lr{N}} dg_x = \int\limits_{\mathbb{H}^{N\times N}} d\phi_x \expa{-N \tr\ln\lr{1 + \alpha^2 \phi_x^2}} .
\end{eqnarray}
We now see that in order to get the canonical form of the free (quadratic) part of the action, we have to use $\alpha^2 = \frac{\lambda}{8}$. Taking (\ref{pcm_action0}) and (\ref{stereo_jacobian}) together, we obtain the following representation of the action:
\begin{eqnarray}
\label{pcm_action}
 S\lrs{\phi_x} =
 \frac{1}{2} \sum\limits_{x,y} \lr{D_{xy} + \frac{\lambda}{4} \delta_{x y}} \tr\lr{\phi_x \phi_y}
 + \nonumber \\ +
 \sum\limits_{k=2}^{+\infty} \lr{-1}^{k-1} \frac{\lambda^{k}}{k \, 8^k} \sum\limits_x \tr \phi_x^{2 k}
 + \nonumber \\ +
 \sum\limits_{\substack{k,l=1\\k+l=2 n,k+l>2}}^{+\infty}
  \lr{-1}^{\frac{k-l}{2}} \frac{4 \, \lambda^{\frac{k+l-2}{2}}}{8^{\frac{k+l}{2}}}
       \sum\limits_{x,y} D_{xy} \tr\lr{\phi_x^k \, \phi_y^l}
\end{eqnarray}
It is convenient to rewrite the interacting part of the action as
\begin{eqnarray}
\label{pcm_action_int}
 S_I\lrs{\phi_x}
 = \sum\limits_{n=2}^{+\infty} \lr{- \alpha^2}^{n-1} S_I^{\lr{2 n}}\lrs{\phi_x} ,
 \nonumber \\
 S_I^{\lr{2 n}}\lrs{\phi_x} =
 \lr{
  \frac{\alpha^2}{n} \sum\limits_x \tr \phi_x^{2 n}
  +
  \frac{1}{2} \sum\limits_{l=1}^{2 n - 1} \lr{-1}^{l-1}
  \sum\limits_{x,y} D_{x y} \tr\lr{\phi_x^{2 n - l} \phi_y^l }
 }
\end{eqnarray}

It is instructive to explicitly write out the terms involving interaction vertices with powers of $\phi$ equal to four and six:
\begin{eqnarray}
\label{interaction_action_phi4}
 S_I^{\lr{4}} = \frac{\lambda}{16} \sum\limits_x \tr \phi_x^4
 +
 \frac{1}{2} \sum\limits_{x, y} D_{x y} \lr{
  2 \tr\lr{\phi_x \phi_y^3}
  -
  \tr\lr{\phi_x^2 \phi_y^2} } \\
\label{interaction_action_phi6}
 S_I^{\lr{6}} = \frac{\lambda}{24} \sum\limits_x \tr \phi_x^6
 + \nonumber \\ +
 \frac{1}{2} \sum\limits_{x, y} D_{x y} \lr{
  2 \tr\lr{\phi_x^5 \phi_y}
  -
  2 \tr\lr{\phi_x^4 \phi_y^2}
  +
  \tr\lr{\phi_x^3 \phi_y^3}
 }
\end{eqnarray}

\section{Schwinger-Dyson equations in coordinate space}
\label{sec:sd_coordinate}

 In order to derive the SD equations, we consider the following full derivative:
\begin{eqnarray}
\label{sd_eqs0}
 \delta_{ik} \delta_{jl} \, \int \mathcal{D}\phi_x \frac{\partial}{\partial \phi_{x_1 \, ij}}
 \lr{
 \lrs{\phi_{x_2} \ldots \phi_{x_n}}_{kl} e^{-S\lrs{\phi_x}}
 } = 0
\end{eqnarray}
From this we obtain for correlators with $n = 2$
\begin{eqnarray}
\label{sd_eqs_G2_coord}
 \delta_{x_1 x_2} - \vev{\phi_{x_2} \frac{\partial S}{\partial \phi_{x_1}} } = 0 .
\end{eqnarray}
and for higher-order correlators with $n > 2$ (in fact, $n \geq 4$)
\begin{eqnarray}
\label{sd_eqs_Gn_coord}
 \delta_{x_1 x_2} \vev{\phi_{x_3} \ldots \phi_{x_n}}
 +
 \delta_{x_1 x_n} \vev{\phi_{x_2} \ldots \phi_{x_{n-1}}}
 + \nonumber \\ +
 \sum\limits_{A=3}^{n-1} \delta_{x_1 x_A} \,
 \vev{\phi_{x_2} \ldots \phi_{x_{A-1}}}
 \vev{\phi_{x_{A+1}} \ldots \phi_{x_n}}
 -
 \vev{\phi_{x_2} \ldots \phi_{x_n} \frac{\partial S}{\partial \phi_{x_1}} } = 0 .
\end{eqnarray}
The derivative of the action can be represented as follows:
\begin{eqnarray}
\label{dSdphi_coord}
 \frac{\partial S\lrs{\phi}}{\partial \phi_x}
 =
 \sum\limits_{y} \lr{D_{xy} + \frac{\lambda}{4} \delta_{x y}} \phi_y
 +
 \sum\limits_{l=1}^{+\infty} \lr{-1}^l \frac{2 \lambda^{l+1}}{8^{l+1}} \phi_x^{2 l + 1}
 + \nonumber \\ +
 \sum\limits_{\substack{k,l=1\\k+l=2 n,k+l>2}}^{+\infty}
 \lr{-1}^{\frac{k-l}{2}} \lr{\frac{\lambda}{8}}^{\frac{k+l-2}{2}}
 \sum\limits_y D_{x y}
 \sum\limits_{m=0}^{k-1}
 \phi_x^m \phi_y^l \phi_x^{k-m-1}
 = \nonumber \\ =
 \sum\limits_y \lrs{G_0^{-1}}_{x y} \, \phi_y
 + \nonumber \\ +
 \sum\limits_{k=1}^{+\infty} \lr{-\alpha^2}^k
 \lr{
  \frac{\lambda}{4} \phi_x^{2 k + 1}
  +
  \sum\limits_{l=1}^{2 k + 1} \lr{-1}^{l-1}
  \sum\limits_{m=0}^{2 k + 1 - l}
  \sum\limits_y D_{x y}
  \phi_x^m \phi_y^l \phi_x^{2 k + 1 - l - m}
 }
\end{eqnarray}

\subsection{Schwinger-Dyson equations in algorithmic form}
\label{subsec:sd_coord_alg}

We now assume the expansion
\begin{eqnarray}
\label{expansion_assumed}
 \vev{x_0 \ldots x_{2 n - 1}} = \sum\limits_{m=0}^{+\infty}
 \vev{x_0 \ldots x_{2 n - 1}}_m \lr{-\alpha^2}^m
\end{eqnarray}

\begin{eqnarray}
\label{sd_g2_coord_algorithmic}
 \vev{x_0 x_1}_m = G_0\lr{x_0, x_1} \delta_{m, 0}
 - \nonumber \\ -
 G_0\lr{x_0, z}
 \sum\limits_{k=1}^{m}
 \left(
 \frac{\lambda}{4} \vev{z^{2 k + 1} x_1}_{m - k}
 + \right. \nonumber \\ \left.
 \sum\limits_{l=1}^{2 k + 1} \lr{-1}^{l-1}
 \sum\limits_{m=0}^{2 k + 1 - l}
 \sum\limits_y D_{z y}
 \vev{z^m y^l z^{2k+1-l-m} x_1}_{m - k}
 \right)
\end{eqnarray}

\begin{eqnarray}
\label{sd_gn_coord_algorithmic}
 \vev{x_0 x_1 \ldots x_{2 n - 1}}_m
 = \nonumber \\ =
 G_0\lr{x_0, x_1} \vev{x_2 x_3 \ldots x_{2 n - 1}}_m
 +
 G_0\lr{x_0, x_{2 n - 1}} \vev{x_1 x_3 \ldots x_{2 n - 2}}_m
 + \nonumber \\ +
 \sum\limits_{A=1}^{n-2}
 G_0\lr{x_0, x_{2A+1}}
 \sum\limits_{m'=0}^m
 \vev{x_1 x_2 \ldots x_{2 A}}_{m'}
 \vev{x_{2 A+2} \ldots x_{2 n - 1}}_{m - m'}
 - \nonumber \\ -
 \sum\limits_{k=1}^{m} \sum\limits_z
 G_0\lr{x_0, z}
 \left(
 \frac{\lambda}{4} \vev{z^{2 k + 1} x_1 \ldots x_{2 n - 1}}_{m - k}
 + \right. \nonumber \\ \left.
 \sum\limits_{l=1}^{2 k + 1} \lr{-1}^{l-1}
 \sum\limits_{m'=0}^{2 k + 1 - l}
 \sum\limits_y D_{z y}
 \vev{z^{m'} y^l z^{2k+1-l-m'} x_1 \ldots x_{2 n - 1}}_{m - k}
 \right)
\end{eqnarray}


\subsection{Correlators of $g_x$ variables}
\label{subsec:gx_correlators_coordinate}

\begin{eqnarray}
\label{gx_vev_coordinate}
 \vev{g_x}_{m_{max}} = 1 + 2 \sum\limits_{n=1}^{m_{max}+1} \sum\limits_{m=0}^{m_{max}-n+1} \lr{- \, \alpha^2}^{n+m} \vev{\phi_x^{2 n}}_m
\end{eqnarray}

\begin{eqnarray}
\label{gx_vev_coordinate}
 \vev{g^{\dag}_0 g_x}_{m_{max}}
 =
 -1 + 2 \vev{g_x}
 + \nonumber \\ +
 4 \sum\limits_{n=1}^{m_{max}+1}
   \sum\limits_{m=0}^{m_{max}-n+1}
 \lr{-\alpha^2}^{n+m}
 \sum\limits_{k=1}^{2 n - 1} \lr{-1}^k
 \vev{\phi_x^k \, \phi_0^{2n - k}}_m
\end{eqnarray}


\subsection{Observables in coordinate space}

\section{Schwinger-Dyson equations in momentum space}
\label{sec:sd_momentum}

 Let us now define the momentum-space field operators and correlators as
\begin{eqnarray}
\label{momentum_space_fields}
 \phi_p = \frac{1}{V} \sum\limits_x e^{i p x} \phi_x,
 \quad
 \phi_x = \sum\limits_p e^{-i p x} \phi_p ,
 \nonumber \\
 \vev{p_1 \ldots p_n} \equiv \vev{\phi_{p_1} \ldots \phi_{p_n}} .
\end{eqnarray}
In terms of the momentum-space fields $\phi_p$, the SD equations (\ref{sd_eqs_G2_coord}) and (\ref{sd_eqs_Gn_coord}) read
\begin{eqnarray}
\label{sd_eqs_G2_mom}
 \vev{\frac{\partial S}{\partial \phi_{p_1}} \phi_{p_2} } = \frac{\delta_{p_1 + p_2}}{V}  .
\end{eqnarray}
and for higher-order correlators with $n > 2$ (in fact, $n \geq 4$)
\begin{eqnarray}
\label{sd_eqs_Gn_mom}
 \vev{\frac{\partial S}{\partial \phi_{p_1}} \phi_{p_2} \ldots \phi_{p_n} }
 =
 \frac{\delta_{p_1 + p_2}}{V} \vev{\phi_{p_3} \ldots \phi_{p_n}}
 +
 \frac{\delta_{p_1 + p_2}}{V} \vev{\phi_{p_2} \ldots \phi_{p_{n-1}}}
 + \nonumber \\ +
 \sum\limits_{A=3}^{n-1} \frac{\delta_{p_1 + p_A}}{V} \,
 \vev{\phi_{p_2} \ldots \phi_{p_{A-1}}}
 \vev{\phi_{p_{A+1}} \ldots \phi_{p_n}} .
\end{eqnarray}
The Fourier transform of the derivative of the action $\frac{\partial S}{\partial \phi_p}$ reads:
\begin{eqnarray}
\label{dSdphi_mom}
 \frac{\partial S\lrs{\phi}}{\partial \phi_p}
 \equiv
 \frac{1}{V} \sum\limits_p e^{i p x} \frac{\partial S\lrs{\phi}}{\partial \phi_x}
 =
 \lr{D\lr{p} + \frac{\lambda}{4}} \phi_p
 - \nonumber \\ -
 \sum\limits_{l=1}^{+\infty} \lr{-1}^{l-1} \frac{2 \lambda^{l+1}}{8^{l+1}}
 \sum\limits_{q_1 \ldots q_{2 l + 1}}
 \delta\lr{p - q_1 - \ldots - q_{2 l + 1}}
 \phi_{q_1} \ldots \phi_{q_{2 l + 1}}
 - \nonumber \\ -
 \sum\limits_{\substack{k,l=1\\k+l=2 n,k+l>2}}^{+\infty}
 \lr{-1}^{\frac{k-l}{2}-1} \lr{\frac{\lambda}{8}}^{\frac{k+l-2}{2}}
 \sum\limits_{m=0}^{k-1}
 \nonumber \\
 \sum\limits_{q_1, \ldots, q_{k-1}}
 \sum\limits_{k_1, \ldots, k_l}
 \delta\lr{p - \bar{q} - \bar{k}} \, D\lr{\bar{k}} \,
 \phi_{q_1} \ldots \phi_{q_m}
 \phi_{k_1} \ldots \phi_{k_l}
 \phi_{q_{m+1}} \ldots \phi_{q_{k-1}} ,
\end{eqnarray}
where we have denoted $\bar{k} = k_1 + \ldots + k_l$ and similarly for $q$.
It is also instructive to write out explicitly the interaction terms with exactly three $\phi$ variables (which correspond to the simplest nontrivial interaction vertices in the diagrammatic rules of the theory):
\begin{eqnarray}
\label{dsdPhi_phi4}
 \frac{\partial S_I^{\lr{4}}\lrs{\phi}}{\partial \phi_p}
 =
 - \frac{\lambda}{8}
 \sum\limits_{q_1, q_2, q_3}
 \delta\lr{p - q_1 - q_2 - q_3} \,
 V\lr{q_1, q_2, q_3} \,
 \phi_{q_1} \phi_{q_2} \phi_{q_3} ,
\end{eqnarray}
where we have defined the ``vertex function'' for the four-momentum vertex
\begin{eqnarray}
\label{vertex_function4}
 V\lr{q_1, q_2, q_3} =
 \left(\frac{\lambda}{4} + D\lr{q_1 + q_2 + q_3} - D\lr{q_1 + q_2}
 - \right. \nonumber \\ \left. -
 D\lr{q_2 + q_3} + D\lr{q_1} + D\lr{q_2} + D\lr{q_3}\right)
\end{eqnarray}
Interestingly, at small momenta, when all the $D\lr{q}$ functions can be approximated by $q^2$, the vertex function reduces to $\frac{\lambda}{4} + \lr{q_1 + q_3}^2$ - which is manifestly positive and \textbf{very good for the sign problem in MC}!!!

\subsection{Vertex functions}
\label{subsec:vertex_func}

 The equation (\ref{dSdphi_mom}) can be also compactly represented in terms of the vertex functions:
\begin{eqnarray}
\label{dSdphi_mom_vertex}
 \frac{\partial S\lrs{\phi}}{\partial \phi_p}
 =
 \lr{D\lr{p} + \frac{\lambda}{4}} \phi_p
 + \nonumber \\ +
 \sum\limits_{n=1}^{+\infty} \lr{-\frac{\lambda}{8}}^{n}
 \sum\limits_{q_1 \ldots q_{2 n + 1}} \delta\lr{p - q_1 - \ldots - q_{2 n + 1}}
 V\lr{q_1, \ldots, q_{2 n + 1}} ,
\end{eqnarray}
where the arbitrary-order vertex functions are defined as
\begin{eqnarray}
\label{vertex_func_def}
 V\lr{q_1, \ldots, q_{2 n + 1}}
 =
 \frac{\lambda}{4} +
 \sum\limits_{l=1}^{2 n + 1} \lr{-1}^{l-1}
 \sum\limits_{m=0}^{2 n + 1 - l}
 D\lr{q_{m+1} + \ldots + q_{m+l}}
\end{eqnarray}
For algorithmic purposes, it is more convenient to use the following representation:
\begin{eqnarray}
\label{vertex_func_def_algorithmic}
 V\lr{q_0, \ldots, q_{2 n}}
 =
 \frac{\lambda}{4} +
 \sum\limits_{m=0}^{2 n}
 \sum\limits_{l=0}^{2 n - m}
 \lr{-1}^{l}
 D\lr{q_{m} + \ldots + q_{m+l}}
\end{eqnarray}

Correspondingly, the SD equations can be written in the following compact form:
\begin{eqnarray}
\label{sd_eqs_G2_mom_vertex}
 \vev{p_1 p_2} = \frac{\delta\lr{p_1 + p_2}}{V} \, G_0\lr{p_1}
 - \nonumber \\ -
 G_0\lr{p_1} \sum\limits_{m=1}^{+\infty}
 \lr{-\frac{\lambda}{8}}^m
 \times \nonumber \\ \times
 \sum\limits_{q_1, \ldots, q_{2 m + 1}}
 \delta\lr{p_1 - Q}
 V\lr{q_1, \ldots, q_{2 m + 1}}
 \vev{q_1 \ldots q_{2 m + 1} \, p_2} .
\end{eqnarray}
and for higher-order correlators with $n > 2$ (in fact, $n \geq 4$)
\begin{eqnarray}
\label{sd_eqs_Gn_mom_vertex}
 \vev{p_1 p_2 \ldots p_n}
 =
 \frac{\delta\lr{p_1 + p_2}}{V} \, G_0\lr{p_1} \, \vev{p_3 \ldots p_n}
 + \nonumber \\ +
 \frac{\delta\lr{p_1 + p_n}}{V} \, G_0\lr{p_1} \, \vev{p_2 \ldots p_{n-1}}
 + \nonumber \\ +
 \sum\limits_{A=4}^{n-2}
 \frac{\delta\lr{p_1 + p_A}}{V} \, G_0\lr{p_1} \,
 \vev{p_2 \ldots p_{A-1}} \, \vev{p_{A+1} \ldots p_n}
 - \nonumber \\ -
 G_0\lr{p_1} \sum\limits_{m=1}^{+\infty}
 \lr{-\frac{\lambda}{8}}^m
 \times \nonumber \\ \times
 \sum\limits_{q_1, \ldots, q_{2 m + 1}}
 \delta\lr{p_1 - Q}
 V\lr{q_1, \ldots, q_{2 m + 1}}
 \vev{q_1 \ldots q_{2 m + 1} \, p_2  \ldots p_n} ,
\end{eqnarray}
where we have denoted $Q = q_1 + \ldots + q_{2 n + 1}$.

\subsection{SD equations in ``Algorithmic'' form}
\label{subsec:sds_algorithmic}

 For the purpose of algorithm development, let us define the formal series in powers of $\alpha$ as:
\begin{eqnarray}
\label{alpha_series_def}
 \vev{p_0 \ldots p_{2 n - 1}}
 =
 \sum\limits_{m=0}^{+\infty}
 \vev{p_0 \ldots p_{2 n - 1}}_m \lr{-\frac{\lambda}{8}}^m
\end{eqnarray}

The SD equations (\ref{sd_eqs_G2_mom_vertex}) and (\ref{sd_eqs_Gn_mom_vertex}) can be now rewritten as the recursive relations between successive coefficients in the series (\ref{alpha_series_def}):
\begin{eqnarray}
\label{sd_eqs_G2_mom_vertex_alg}
 \vev{p_0 p_1}_m = \frac{\delta\lr{p_0 + p_1}}{V} \, G_0\lr{p_0} \, \delta_{m,0}
 - \nonumber \\ -
 \lr{1 - \delta_{m,0}} \,
 G_0\lr{p_0} \sum\limits_{k=1}^{m}
 \times \nonumber \\ \times
 \sum\limits_{q_0, \ldots, q_{2 k}}
 \delta\lr{p_0 - Q}
 V\lr{q_0, \ldots, q_{2 k}}
 \vev{q_0 \ldots q_{2 k} \, p_1}_{m-k} .
\end{eqnarray}
and for higher-order correlators with $n > 1$
\begin{eqnarray}
\label{sd_eqs_Gn_mom_vertex_alg}
 \vev{p_0 p_1 \ldots p_{2n-1}}_m
 =
 \frac{\delta\lr{p_0 + p_1}}{V} \, G_0\lr{p_0} \, \vev{p_2 \ldots p_{2n-1}}_m
 + \nonumber \\ +
 \frac{\delta\lr{p_0 + p_{2n-1}}}{V} \, G_0\lr{p_0} \, \vev{p_1 \ldots p_{2n-2}}_m
 + \nonumber \\ +
 \sum\limits_{A=1}^{n-2}
 \frac{\delta\lr{p_0 + p_{2A+1}}}{V} \, G_0\lr{p_0} \,
 \sum\limits_{k=0}^m
 \vev{p_1 \ldots p_{2A}}_k \, \vev{p_{2 A+2} \ldots p_{2n-1}}_{m-k}
 - \nonumber \\ -
 \lr{1 - \delta_{m,0}} \,
 G_0\lr{p_0} \sum\limits_{k=1}^{m}
 \times \nonumber \\ \times
 \sum\limits_{q_0, \ldots, q_{2 k}}
 \delta\lr{p_0 - Q}
 V\lr{q_0, \ldots, q_{2 k}}
 \vev{q_0 \ldots q_{2 k} \, p_1  \ldots p_{2n-1}}_{m-k} ,
\end{eqnarray}
where we have denoted $Q = q_0 + \ldots + q_{2 k}$.

\subsection{Lowest-order results with explicit formulae}

 Plugging the equation (\ref{dsdPhi_phi4}) into the lowest-order SD equation (\ref{sd_eqs_G2_mom}), we get for the first order nontrivial correction to the Green's function:
\begin{eqnarray}
\label{p0p1_1}
 \vev{p_0 p_1}_1 = G_0^2\lr{p_0} \frac{\delta\lr{p_0 + p_1}}{V} \Sigma_1\lr{p_0} ,
\end{eqnarray}
where
\begin{eqnarray}
\label{sigma_1}
 \Sigma_1\lr{p} = -\frac{1}{V} \sum\limits_q \lr{V\lr{q, -q, p} + V\lr{p, q, -q}} G_0\lr{q}
 = \nonumber \\ =
 -\frac{2}{V} \sum\limits_q \lr{\frac{\lambda}{4} + 2 D\lr{p} + 2 D\lr{q} - D\lr{p-q}} G_0\lr{q} .
\end{eqnarray}
In particular, $\Sigma_1\lr{0} = -2$. If we interpret this quantity as the first-order correction to the self-energy, then $\Sigma_1\lr{0}$ exactly cancels the bare mass term $\lambda/4$!

 The lowest-order result for $\vev{k_0 k_1 k_2 k_3 k_4 k_5}$ can be written as
\begin{eqnarray}
\label{G6_0}
 \vev{k_0 k_1 k_2 k_3 k_4 k_5}_0
 =
 \frac{1}{V^3} G_0\lr{k_0} G_0\lr{k_2} G_0\lr{k_4}
 \times \nonumber \\ \times
 \left(
 \delta\lr{k_0 + k_1} \delta\lr{k_2 + k_3} \delta\lr{k_4 + k_5}
 \right. + \red{(u1)} \, \nonumber\\ +
 \delta\lr{k_0 + k_1} \delta\lr{k_2 + k_5} \delta\lr{k_4 + k_3}
         + \red{(u2)} \, \nonumber\\ +
 \delta\lr{k_0 + k_3} \delta\lr{k_2 + k_1} \delta\lr{k_4 + k_5}
         + \red{(u3)} \, \nonumber\\ +
 \delta\lr{k_0 + k_5} \delta\lr{k_2 + k_1} \delta\lr{k_4 + k_3}
         + \red{(u4)} \, \nonumber\\ + \left.
 \delta\lr{k_0 + k_5} \delta\lr{k_2 + k_3} \delta\lr{k_4 + k_1}
           \red{(u5)}
 \right)
\end{eqnarray}
From this we can infer the leading-order correction to $\vev{l_0 l_1 l_2 l_3}$:
\begin{eqnarray}
\label{G4_1}
 \vev{l_0 l_1 l_2 l_3}_1
 = \nonumber \\ =
  \frac{\delta\lr{l_0 + l_1}}{V} \, \frac{\delta\lr{l_2 + l_3}}{V} \lr{ G_0\lr{l_0} G_1\lr{l_2} + G_1\lr{l_0} G_0\lr{l_2} }
  + \nonumber \\ +
  \frac{\delta\lr{l_0 + l_3}}{V} \, \frac{\delta\lr{l_1 + l_2}}{V} \lr{ G_0\lr{l_0} G_1\lr{l_2} + G_1\lr{l_0} G_0\lr{l_2} }
 - \nonumber \\ -
 \frac{\delta\lr{l_0 + l_1 + l_2 + l_3}}{V^3}
 G_0\lr{l_0} G_0\lr{l_1} G_0\lr{l_2} G_0\lr{l_3}
 V\lr{l_1, l_2, l_3} ,
\end{eqnarray}
where we have defined $G_1\lr{p} = G_0\lr{p} \Sigma_1\lr{p} G_0\lr{p}$.

With this result, we can find
\begin{eqnarray}
\label{G2_2}
 \vev{p_0 p_1}_2
 = \nonumber \\ =
 \frac{\delta\lr{p_0 + p_1}}{V} G_0\lr{p_0} \Sigma_1\lr{p_0} G_1\lr{p_0}
 +
 \frac{\delta\lr{p_0 + p_1}}{V} G_0\lr{p_0} \Sigma_2\lr{p_0} G_0\lr{p_0}
 ,
\end{eqnarray}
where $\Sigma_2\lr{p} = \Sigma_2^a\lr{p} + \Sigma_2^b\lr{p} + \Sigma_2^c\lr{p}$ and
\begin{eqnarray}
\label{sigma2_a}
 \Sigma_2^a\lr{p} = - \frac{1}{V} \sum\limits_q \lr{V\lr{q, -q, p} + V\lr{p, q, -q}} G_1\lr{q}
\\
\label{sigma2_b}
 \Sigma_2^b\lr{p} = \frac{1}{V^2} \sum\limits_{q_1, q_2}
 V\lr{p - q_1 - q_2, q_1, q_2}
 \times \nonumber \\ \times
 G_0\lr{p - q_1 - q_2} G_0\lr{q_1} G_0\lr{q_2}
 V\lr{-q_1, -q_2, -p_1}
\\
\label{sigma2_c}
 \Sigma_2^c\lr{p} = \frac{1}{V^2} \sum\limits_{q_1, q_2}
 G\lr{q_1} G\lr{q_2}
 \left(
  V\lr{-q_1, q_1, q_2, -q_2, -p}
  + \right. \nonumber \\ +
  V\lr{-q_1, q_1, -p, -q_2, q_2}
  + \nonumber \\ +
  V\lr{-q_2, q_1, -q_1, q_2, -p}
  + \nonumber \\ +
  V\lr{p_0, q_1, -q_1, q_2, -q_2}
  + \nonumber \\ \left. +
  V\lr{p_0, q_1, q_2, -q_2, -q_1}
 \right)
 = \nonumber \\ =
 \frac{1}{V^2} \sum\limits_{q_1, q_2}
 G\lr{q_1} G\lr{q_2}
 \times \nonumber \\ \times
 \left(
 \frac{5 \lambda }{4}
 +14 D\lr{p}
  -6 D\lr{p-q_1}
  +14 D\lr{q_1}
 - \right. \nonumber \\ -
 3 D\lr{p-q_2}
 - 3 D\lr{q_1-q_2}
  + D\lr{p+q_1-q_2}
  +14 D\lr{q_2}
  - \nonumber \\ \left. -
  3 D\lr{p+q_2}
  +D\lr{p-q_1+q_2}
  -3 D\lr{q_1+q_2}
   +D\lr{-p+q_1+q_2}
 \right)
 = \nonumber \\ =
 \lr{\frac{5 \lambda}{4} + 14 D\lr{p}} I_0^2
 - 12 I_0 \vev{G_0\lr{q} D\lr{p - q}}
 + \nonumber \\ +
 28 I_0 \lr{1 - \frac{\lambda}{4} I_0}
 - 6 \vev{D\lr{q_1 - q_2} G_0\lr{q_1} G_0\lr{q_2}}
 + \nonumber \\ +
  3 \vev{D\lr{p - q_1 - q_2} G_0\lr{q_1} G_0\lr{q_2}}
\end{eqnarray}
For some particular terms in this sum we find
\begin{eqnarray}
\label{s2c_term1}
 \vev{G_0\lr{q} D\lr{p - q}} = \lr{1 - \frac{\lambda}{4} I_0} \lr{1 - \frac{1}{2 d} D\lr{p}}
 + D\lr{p} I_0
\end{eqnarray}

\subsubsection{Some integrals in 1D and 2D...}

For the infinite 1D lattice,
\begin{eqnarray}
\label{phixphiy0}
 \vev{\phi_x \phi_y}_0 = \int\limits_{-\pi}^{\pi} \frac{d p}{2 \pi} \, \frac{\cos\lr{p\lr{x - y}}}{\frac{\lambda}{4} + 4 \sin^2\lr{\frac{p}{2}}}
 \nonumber \\
 \vev{\phi_0 \phi_0}_0 \equiv I_0\lr{\lambda} =  \int\limits_{-\pi}^{\pi} \frac{d p}{2 \pi} \, \frac{1}{\frac{\lambda}{4} + 4 \sin^2\lr{\frac{p}{2}}} = \frac{4}{\sqrt{\lambda \, \lr{16 + \lambda}}}
 \nonumber \\
 \vev{\phi_0 \phi_1}_0 =  \int\limits_{-\pi}^{\pi} \frac{d p}{2 \pi} \, \frac{1 - 2 \sin^2\lr{\frac{p}{2}}}{\frac{\lambda}{4} + 4 \sin^2\lr{\frac{p}{2}}} =
 \lr{1 + \frac{\lambda}{8}} I_0\lr{\lambda} - \frac{1}{2}
\end{eqnarray}
We can now explicitly find the self-energy $\Sigma_1\lr{p}$:
\begin{eqnarray}
\label{sigma1p}
 -\frac{1}{2} \Sigma_1\lr{p}
 =
 1 + \frac{D\lr{p}}{2}\lr{1 + \lr{2 - \frac{\lambda}{4}} I_0\lr{\lambda} }
\end{eqnarray}

For the infinite 2D lattice,
\begin{eqnarray}
\label{I0_2D}
 I_0\lr{\lambda} = \frac{8 \text{EllipticK}\left[-\frac{256}{32 \lambda +\lambda ^2}\right]}{\pi  \sqrt{\lambda  (32+\lambda )}}
\end{eqnarray}
We can now explicitly find the self-energy $\Sigma_1\lr{p}$:
\begin{eqnarray}
\label{sigma1p}
 -\frac{1}{2} \Sigma_1\lr{p}
 =
 1 + \frac{D\lr{p}}{4}\lr{1 + \lr{4 - \frac{\lambda}{4}} I_0\lr{\lambda} }
\end{eqnarray}

For the expectation values of the $g_x$ operators, to the lowest orders in $\lambda$ we have
\begin{eqnarray}
\label{gx_vev_lowest}
 \vev{g_x} = 1 - \frac{\lambda}{4} \vev{\phi_x^2} + \frac{\lambda^2}{32} \vev{\phi_x^4}
\end{eqnarray}

\begin{eqnarray}
\label{gxgy_vev_lowest}
 \vev{g^{\dag}_x g_y} =
 -1 + 2 \vev{g_x} + \frac{\lambda}{2} \vev{\phi_x \phi_y}
 + \nonumber \\ +
   \frac{\lambda^2}{16} \vev{\phi_x^2 \phi_y^2}
 - \frac{\lambda^2}{16} \lr{\vev{\phi_x \phi_y^3} + \vev{\phi_x^3 \phi_y}}
\end{eqnarray}

We can now calculate
\begin{eqnarray}
\label{gcorrs_order0}
 \vev{g_0}_0 = 1 - \frac{\lambda}{4} I_0\lr{\lambda}
 \nonumber \\
 \vev{g^{\dag}_0 g_1} = 1 - \frac{\lambda}{4} + \frac{\lambda^2}{16} I_0\lr{\lambda} = 1 - \frac{\lambda}{4} + \frac{\lambda^{3/2}}{16} + O\lr{\lambda^2}
\end{eqnarray}

\begin{eqnarray}
\label{}
 \vev{g^{\dag}_p g_q}_1 =
 \lr{2 \vev{g} - 1} \delta\lr{p} \delta\lr{q}
 +
 \frac{\lambda}{2}
 \lr{\vev{p q}_0 - \frac{\lambda}{8} \vev{p q}_1}
 + \nonumber \\ +
 \frac{\lambda^2}{16} \frac{-4 \delta\lr{p+q}}{V} G_0\lr{p} \frac{1}{V} \sum\limits_l G_0\lr{l}
 + \frac{\lambda^2}{16} \delta\lr{p} \delta\lr{q} \lr{\frac{1}{V} \sum\limits_l G_0\lr{l}}^2
 + \nonumber \\ +
 + \frac{\lambda^2}{16} \frac{\delta\lr{p+q}}{V} \frac{1}{V} \sum\limits_{l} G_0\lr{l} G_0\lr{q+l}
 \end{eqnarray}

\subsection{Correlators of $g_x$ variables}
\label{subsec:gx_correlators_momentum}

 An important criterion of whether the $SU\lr{N} \times SU\lr{N}$ symmetry of the action is restored dynamically is the expectation value of a single field variable $\vev{g_x}$, which should tend to zero as the approximation becomes more and more exact. Expressing $\vev{g_x}$ in terms of the variables $\phi_x$, we get:
\begin{eqnarray}
\label{gx_vev_momentum}
 \vev{g_x} \equiv \frac{1}{V} \sum\limits_x \vev{g_x}
 =
 1 + 2 \sum\limits_{k=1}^{+\infty} \lr{-\frac{\lambda}{8}}^k \frac{1}{V} \sum\limits_x \vev{\phi_x^{2 k}}
 = \nonumber \\ =
 1 + 2 \sum\limits_{k=1}^{+\infty} \lr{-\frac{\lambda}{8}}^k
 \sum\limits_{q_1 \ldots q_{2 k}}
 \delta\lr{q_1 + \ldots + q_{2 k}}
 \vev{\phi_{q_1} \ldots \phi_{q_{2k}}} .
\end{eqnarray}
In the same way, we can express the correlator $\vev{g^{\dag}_x g_y}$ in coordinate space as:
\begin{eqnarray}
\label{gxgy_vev_space}
 \vev{g^{\dag}_x g_y}
 =
 2 \vev{g_x} - 1
 +
 4 \sum\limits_{k,l=1}^{+\infty}
 \lr{-1}^{\frac{k-l}{2}} \, \lr{\frac{\lambda}{8}}^{\frac{k+l}{2}}
 \vev{\phi_x^k \phi_y^l}
 = \nonumber \\ =
 \lr{2 \vev{g_x} - 1} + 4 \sum\limits_{n=1}^{+\infty} \lr{-\frac{\lambda}{8}}^n
 \sum\limits_{k=1}^{2 n - 1} \lr{-1}^k \vev{\phi_x^k \phi_y^{2 n - k}}
\end{eqnarray}
It is also convenient to write out the result for the Fourier transform of $\vev{g^{\dag}_x g_y}$:
\begin{eqnarray}
\label{gxgy_vev_fourier}
 \vev{g^{\dag}_p g_q}
 =
 \lr{2 \vev{g_x} - 1} \delta\lr{p} \delta\lr{q}
 + \nonumber \\ +
 4 \sum\limits_{m=1}^{+\infty} \lr{-\frac{\lambda}{8}}^m
 \sum\limits_{p_0 \ldots p_{2 m-1}}
 \sum\limits_{l = 1}^{2 m - l} \lr{-1}^l
 \delta\lr{p - p_0 - \ldots - p_{l-1}}
 \delta\lr{q - p_l - \ldots - p_{2 m - 1}}
 \vev{p_0 \ldots p_{2 m-1}}
\end{eqnarray}


Performing the Fourier transform and assuming translational invariance, we get
\begin{eqnarray}
\label{gx_vev_momentum}
 \vev{g^{\dag}_x g_0}
 =
 2 \vev{g_x} - 1
 + \nonumber \\ +
 4 \sum\limits_{n=1}^{+\infty} \lr{-\lambda/8}^n
 \sum\limits_{q_1 \ldots q_{2 n}}
 \Gamma\lr{x; \, q_1, \ldots, q_{2 n}}
 \vev{\phi_{q_1} \ldots \phi_{q_{2 n}}} ,
 \nonumber \\
 \Gamma\lr{x; \, q_1, \ldots, q_{2 n}} =
 \sum\limits_{k=1}^{2 n - 1} \lr{-1}^k \cos\lr{\lr{q_1 + \ldots + q_k} x}
\end{eqnarray}

\section{SD metropolis for our model}

\subsection{Estimating the maximal weight of the vertex action}

The maximum of $V$, as a short numerical experiment shows, can be found by setting all the momenta to $\pm \pi$, the corner of the Brillouin zone. This way we obtain, in $D=1$:
\begin{eqnarray}
\label{vertex_max}
 \max{V\lr{q_1, \ldots, q_{2m+1}}} = 4 \lr{m+1}^2
\end{eqnarray}
Thus the maximal probability of the ``vertex'' action is (in the case of infinite number of momenta, all of which are equal to $\pi$:)
\begin{eqnarray}
\label{vertex_max_prob}
 \sum\limits_{m=1}^{+\infty} c^m 4 \lr{m+1}^2 =
\end{eqnarray}

\subsection{Setting the values for $c$ and $\mathcal{N}$ in free planar field theory}

The normalization of the free propagators involves the Catalan numbers:
\begin{eqnarray}
\label{free_greenfunc_normalizations}
 \sum\limits_{\lrc{p}} \vev{p_1 \ldots p_{2n}} = \Sigma^n \, \frac{\lr{2 n}!}{n! \lr{n+1}!}
\end{eqnarray}

Now the total sum of probabilities over $n$ is
\begin{eqnarray}
\label{free_greenfunc_normalizations}
 \sum\limits_{n=1,\lrc{p}}^{+\infty} c^{-n} \vev{p_1 \ldots p_{2n}}
 =
 \sum\limits_{n=1}^{+\infty}
 \lr{\frac{\Sigma}{c}}^n \, \frac{\lr{2 n}!}{n! \lr{n+1}!}
 \equiv \mathcal{U}\lr{\Sigma/c} =
 \frac{1 - \sqrt{1 - \frac{4 \Sigma}{c}}}{\frac{2 \Sigma}{c}} - 1 ,
\end{eqnarray}
where we have used the well-known expression for the generating function of the Catalan numbers. In particular, this expression sets the limits $c > 4 \Sigma$. It is also obvious that the sum over multiple-trace correlators is only finite if $\mathcal{N} > \mathcal{U}\lr{\Sigma/c} $.

It is now easy to find that the average length of the topmost (or any other) element in the sequence of sequences is $x \partial_x \ln \mathcal{U} = \lr{1 - 4 x}^{-1/2}$, with $x = \Sigma/c$, and the average number of sequences in the stack is $\frac{1}{1 - \mathcal{U}/\mathcal{N}}$.

\section{Cross-check with Gross-Witten higher correlators...}

In Gross-Witten,
\begin{eqnarray}
 \vev{g}   = 1 - \frac{\lambda_{GW}}{4} = 1 - \frac{\lambda}{8}
 \nonumber \\
 \vev{g^2} = 1 - \lambda_{GW} + \frac{\lambda_{GW}^2}{4} = 1 - \frac{\lambda}{2} + \frac{\lambda^2}{16}
\nonumber \\
 \vev{g^3} = 1 - \frac{9 \lambda_{GW}}{4} + \frac{3 \lambda_{GW}^2}{2} - \frac{5 \lambda_{GW}^3}{16} = 1 - \frac{9 \lambda}{8} + \frac{3 \lambda^2}{8} - \frac{5 \lambda^3}{128}
\end{eqnarray}

\newpage
\begin{landscape}
\begin{eqnarray}
 \hspace{-4cm} \sum\limits_{q_1 q_2 q_3 q_4} V\lr{p_0 - q_1 - q_2 - q_3 - q_4, q_1, q_2, q_3, q_4}
 G_0\lr{(p_0 - q_1 - q_2 - q_3 - q_4)} G_0\lr{q_2} G_0\lr{q_4} \delta\lr{p_0 - q_2 - q_3 - q_4} \delta\lr{q_2 + q_3} \delta\lr{q_4 + p_1}
     + \red{(u1)} \, \nonumber\\ \hspace{-4cm} +
 \sum\limits_{q_1 q_2 q_3 q_4} V\lr{p_0 - q_1 - q_2 - q_3 - q_4, q_1, q_2, q_3, q_4}
 G_0\lr{(p_0 - q_1 - q_2 - q_3 - q_4)} G_0\lr{q_2} G_0\lr{q_4} \delta\lr{p_0 - q_2 - q_3 - q_4} \delta\lr{q_2 + p_1} \delta\lr{q_4 + q_3}
         + \red{(u2)} \, \nonumber\\ \hspace{-4cm} +
\sum\limits_{q_1 q_2 q_3 q_4} V\lr{p_0 - q_1 - q_2 - q_3 - q_4, q_1, q_2, q_3, q_4}
 G_0\lr{(p_0 - q_1 - q_2 - q_3 - q_4)} G_0\lr{q_2} G_0\lr{q_4} \delta\lr{p_0 - q_1 - q_2 - q_4} \delta\lr{q_2 + q_1} \delta\lr{q_4 + p_1}
         + \red{(u3)} \, \nonumber\\ \hspace{-4cm} +
\sum\limits_{q_1 q_2 q_3 q_4} V\lr{p_0 - q_1 - q_2 - q_3 - q_4, q_1, q_2, q_3, q_4}
  G_0\lr{(p_0 - q_1 - q_2 - q_3 - q_4)} G_0\lr{q_2} G_0\lr{q_4} \delta\lr{(p_0 - q_1 - q_2 - q_3 - q_4) + p_1} \delta\lr{q_2 + q_1} \delta\lr{q_4 + q_3}
         + \red{(u4)} \, \nonumber\\ \hspace{-4cm} +
  \sum\limits_{q_1 q_2 q_3 q_4} V\lr{p_0 - q_1 - q_2 - q_3 - q_4, q_1, q_2, q_3, q_4}
  G_0\lr{(p_0 - q_1 - q_2 - q_3 - q_4)} G_0\lr{q_2} G_0\lr{q_4} \delta\lr{(p_0 - q_1 - q_2 - q_3 - q_4) + p_1} \delta\lr{q_2 + q_3} \delta\lr{q_4 + q_1}
           \red{(u5)}
\end{eqnarray}
\end{landscape}

\end{document}

\comment{
 \begin{eqnarray}
\label{G4_1_contribs}
 \sum\limits_{q_1, q_2}
 V\lr{l_0 - q_1 - q_2, q_1, q_2} G_0\lr{l_0 - q_1 - q_2} G_0\lr{q_2} G_0\lr{l_2} %u1
 \delta\lr{l_0 - q_1 - q_2 + q_1} \delta\lr{q_2 + l_1} \delta\lr{l_2 + l_3}
 \nonumber \\
 \sum\limits_{q_1, q_2}
 V\lr{l_0 - q_1 - q_2, q_1, q_2} G_0\lr{l_0 - q_1 - q_2} G_0\lr{q_2} G_0\lr{l_2} %u2
 \delta\lr{l_0 - q_1 - q_2 + q_1} \delta\lr{q_2 + l_3} \delta\lr{l_2 + l_1}
 \nonumber \\
 \sum\limits_{q_1, q_2}
 V\lr{l_0 - q_1 - q_2, q_1, q_2} G_0\lr{l_0 - q_1 - q_2} G_0\lr{q_2} G_0\lr{l_2} %u3
 \delta\lr{l_0 - q_1 - q_2 + l_1} \delta\lr{q_2 + q_1} \delta\lr{l_2 + l_3}
 \nonumber \\
 \sum\limits_{q_1, q_2}
 V\lr{l_0 - q_1 - q_2, q_1, q_2} G_0\lr{l_0 - q_1 - q_2} G_0\lr{q_2} G_0\lr{l_2} %u4
 \delta\lr{l_0 - q_1 - q_2 + l_3} \delta\lr{q_2 + q_1} \delta\lr{l_2 + l_1}
 \nonumber \\
 \sum\limits_{q_1, q_2}
 V\lr{l_0 - q_1 - q_2, q_1, q_2} G_0\lr{l_0 - q_1 - q_2} G_0\lr{q_2} G_0\lr{l_2} %u5
 \delta\lr{l_0 - q_1 - q_2 + l_3} \delta\lr{q_2 + l_1} \delta\lr{l_2 + q_1}
\end{eqnarray}
This translates into
\begin{eqnarray}
\label{G4_1_contribs_simplified}
 \sum\limits_{q_1}
 V\lr{- q_1, q_1, l_0} G_0\lr{-q_1} G_0\lr{l_0} G_0\lr{l_2} %u1
 \delta\lr{l_0 + l_1} \delta\lr{l_2 + l_3} %DONE
 \nonumber \\
 \sum\limits_{q_1}
 V\lr{- q_1, q_1, l_0} G_0\lr{-q_1} G_0\lr{l_0} G_0\lr{l_2} %u2
 \delta\lr{l_0 + l_3} \delta\lr{l_2 + l_1} %DONE
 \nonumber \\
 \sum\limits_{q_1}
 V\lr{l_0, q_1, -q_1} G_0\lr{l_0} G_0\lr{q_1} G_0\lr{l_2} %u3
 \delta\lr{l_0 + l_1} \delta\lr{l_2 + l_3} %DONE
 \nonumber \\
 \sum\limits_{q_1}
 V\lr{l_0, q_1, -q_1} G_0\lr{l_0} G_0\lr{q_1} G_0\lr{l_2} %u4
 \delta\lr{l_0 + l_3} \delta\lr{l_2 + l_1}
 \nonumber \\
 V\lr{-l_3, -l_2, -l_1} G_0\lr{l_3} G_0\lr{l_1} G_0\lr{l_2} %u5
 \delta\lr{l_0 + l_2 + l_1 + l_3}
\end{eqnarray}
}

\comment{
\begin{eqnarray}
\label{simplify1}
 -\delta\lr{p - p_0} \delta\lr{q - p_1 - p_2 - p_3}
  \delta\lr{p_0 + p_1} G_0\lr{p_0} \delta\lr{p_2 + p_3} G_0\lr{p_2}
 \nonumber \\
 +\delta\lr{p - p_0 - p_1} \delta\lr{q - p_2 - p_3}
  \delta\lr{p_0 + p_1} G_0\lr{p_0} \delta\lr{p_2 + p_3} G_0\lr{p_2}
 \nonumber \\
 -\delta\lr{p - p_0 - p_1 - p_2} \delta\lr{q - p_3}
  \delta\lr{p_0 + p_1} G_0\lr{p_0} \delta\lr{p_2 + p_3} G_0\lr{p_2}
 \nonumber \\
 -\delta\lr{p - p_0} \delta\lr{q - p_1 - p_2 - p_3}
  \delta\lr{p_0 + p_3} G_0\lr{p_0} \delta\lr{p_2 + p_1} G_0\lr{p_2}
 \nonumber \\
 +\delta\lr{p - p_0 - p_1} \delta\lr{q - p_2 - p_3}
  \delta\lr{p_0 + p_3} G_0\lr{p_0} \delta\lr{p_2 + p_1} G_0\lr{p_2}
 \nonumber \\
 -\delta\lr{p - p_0 - p_1 - p_2} \delta\lr{q - p_3}
  \delta\lr{p_0 + p_3} G_0\lr{p_0} \delta\lr{p_2 + p_1} G_0\lr{p_2}
 \nonumber \\
\end{eqnarray}
}
