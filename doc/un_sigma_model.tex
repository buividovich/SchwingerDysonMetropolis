\documentclass[twocolumn,showpacs,preprintnumbers,superscriptaddress,amsmath,floatfix,amssymb,secnumarabic]{revtex4}
\usepackage[colorlinks=true]{hyperref}
%\usepackage[colorlinks=false]{hyperref}
\usepackage{graphicx}

\newcommand{\comment}[1]{}

\newcommand{\lr}[1]{ \left( #1 \right) }
\newcommand{\lrs}[1]{ \left[ #1 \right] }
\newcommand{\lrc}[1]{ \left\{ #1 \right\} }
\newcommand{\vev}[1]{ \langle \, #1 \, \rangle }

\newcommand{\Tr}{ {\rm Tr} \, }
\newcommand{\tr}{ {\rm Tr} \, }
\newcommand{\re}{ {\rm Re} \, }
\newcommand{\im}{ {\rm Im} \, }
\renewcommand{\Re}{ {\rm Re} \, }
\renewcommand{\Im}{ {\rm Im} \, }

\newcommand{\rvac}{ \, | 0 \rangle }
\newcommand{\lvac}{ \langle 0 | \, }
\newcommand{\ket}[1]{ \, | #1 \rangle }
\newcommand{\bra}[1]{ \langle #1 | \, }

\newcommand{\diag}[1]{ {\rm diag} \, \left( #1 \right) }
\newcommand{\const}{ {\rm const}}
\renewcommand{\det}[1]{ {\rm det} \left( #1 \right) }

\newcommand{\sign}{ {\rm sign} \,  }
\newcommand{\sh}{ {\rm sh} \,  }
\newcommand{\ch}{ {\rm ch} \,  }
\renewcommand{\th}{ {\rm th} \,  }
\newcommand{\hodge}{{}^{*}}
\newcommand{\expa}[1]{ \exp{\left( #1 \right)} }
\newcommand{\abs}[1]{| #1 |}
\newcommand{\zvar}[1]{ e^{\frac{2 \pi i}{N} \, #1  } }
\newcommand{\zvarbr}[1]{ \exp{ \left( \frac{2 \pi i}{N} \left( #1 \right) \right) }}

\newcommand{\red}[1]{\textcolor[rgb]{1.00,0.00,0.00}{#1}}

\newcommand{\logo}{\\ \vskip -18mm
\leftline{\includegraphics[scale=0.3,clip=false]{logo.eps}} \vskip 10mm}

% PACS: 02.70.-c Computational techniques; simulations
%       02.50.Ey Stochastic processes
%       11.15.Pg Expansions for large numbers of components e.g., 1/Nc expansions


\begin{document}
\sloppy

%\logo at the end of the title
\title{Numerical solution of Schwinger-Dyson equations in large-$N$ $U\lr{N}$ sigma model}

\author{P. V. Buividovich}
\email{pavel.buividovich@physik.uni-regensburg.de}
\affiliation{Regensburg University, Regensburg, Germany}

\date{June 19, 2014}
\begin{abstract}
 We propose a numerical stochastic algorithm for solving the Schwinger-Dyson equations in large-$N$ $U\lr{N}$ sigma model.
\end{abstract}
\pacs{02.70.-c; 02.50.Ey; 11.15.Pg}

\maketitle

\section*{Definitions}

For further convenience, let us introduce the notation
\begin{eqnarray}
\label{vev_definition}
 \vev{F\lr{\phi}} = \mathcal{Z}^{-1} \int\limits_{\mathbb{C}^{N \times N}} \mathcal{D}\phi e^{-S\lrs{\Phi}} \frac{1}{N} \tr F\lr{\phi}
\end{eqnarray}
for the expectation value of the normalized trace of some matrix-valued observable $F\lrs{\phi}$, where $\phi$ is some $N \times N$ matrix-valued field.

\section*{Unitarity via (implicit) matrix Lagrange multiplier}

 We start with the partition function of the form
\begin{eqnarray}
\label{sm_partition}
 \mathcal{Z} = \int\limits_{U\lr{N}} dg_x \expa{-\frac{N}{\lambda} \sum\limits_{x,y} D_{xy} \tr\lr{g^{\dag}_x g_y} } ,
\end{eqnarray}
where $D_{xy} = 2 D \delta_{x y} - \sum\limits_{\mu} \delta_{x,y+\hat{\mu}} - \sum\limits_{\mu} \delta_{x,y-\hat{\mu}}$ is the lattice discretization of the continuum operator $-\partial_{\mu}^2$. It is important that this operator is positive-semidefinite. Its eigenvalues are $\lambda \sim \sum\limits_{\mu} 4 \sin^2\lr{k_{\mu}/2}$.

Let us now replace integration over unitary matrices $g_x$ by integration over the full $\mathbb{C}^{N \times N}$ and enforce the unitarity condition $g^{\dag}_x g_x$ by an integral over a hermitian-matrix-valued Lagrange multiplier $\xi_x$:
\begin{widetext}
\begin{eqnarray}
\label{sm_partition_matrix_lagrange}
 \mathcal{Z} = \int dg_x \int d\xi_x \expa{ - \frac{N}{\lambda} \sum\limits_{x \neq y} D_{xy} \tr\lr{g^{\dag}_x g_y} - \frac{i N}{\lambda} \sum\limits_x \tr\lr{\xi_x  g^{\dag}_x g_x - \xi_x} }
\end{eqnarray}
Performing the Gaussian integral over $g_x$ and $g^{\dag}_x$, we arrive at the following representation:
\begin{eqnarray}
\label{sm_partition_matrix_lagrange_only_imaginary}
 \mathcal{Z} = \int d\xi_x
 \expa{N \tr\ln\lr{D_{xy} + i \xi_x \delta_{xy}} + \frac{i N}{\lambda} \sum\limits_x \tr \xi_x } .
\end{eqnarray}
In order to make this path integral over Hermitian matrices well-defined, we now analytically continue the integration variable to $\phi_x = i \xi_x$, assuming again that $\phi_x$ is Hermitian. Simultaneously, we have to change $\lambda \rightarrow -\lambda$. Such a prescription is motivated by a nontrivial structure of Lefshetz thimbles in this theory. This leads, finally, to the following representation of the partition function:
\begin{eqnarray}
\label{sm_partition_matrix_lagrange_only}
 \mathcal{Z} = \int d\phi_x
 \expa{N \tr\ln\lr{D_{xy} + \phi_x \delta_{xy}} - \frac{N}{\lambda} \sum\limits_x \tr \phi_x } .
\end{eqnarray}
\end{widetext}

 The observables like $\vev{g^{\dag}_x g_y }$ are now expressed in terms of $\vev{G_{xy}}$ where $G_{xy}$ is the inverse of the operator $D_{xy} + \phi_x \delta_{xy}$ (which acts on the space of $N$-component fields $\psi_{x i}$). In particular, $\vev{g^{\dag}_x g_y } = \vev{\lambda \, G_{xy}}$. By definition, $G_{xy}$ satisfy the identities of the form
\begin{eqnarray}
\label{G_identities}
 \sum\limits_z D_{xz} G_{zy} + \phi_x G_{xy} = \delta_{x y} ,
 \nonumber \\
 \sum\limits_z G_{xz} D_{zy} + G_{xy} \phi_y = \delta_{x y} .
\end{eqnarray}
Note that these are now the operator identities which hold on configuration per configuration basis.

 Schwinger-Dyson equations for the path integral (\ref{sm_partition_matrix_lagrange_only}) can be derived in terms of several sets of observables. One could, for example, consider the correlators of the $\phi_x$ variable itself. This choice, however, leads to many nonlocal terms in the action. A local form of SD equations can be obtained in terms of the observables of the form $\vev{G_{x_1 y_1} \ldots G_{x_n y_n}}$.

 In order to derive the Schwinger-Dyson equations for the above observables, let us consider the following full derivative:
\begin{widetext}
\begin{eqnarray}
\label{sd_derivation1}
 \delta_{ik} \delta_{jl} \, \int\mathcal{D}\phi_x \frac{\partial}{\partial \phi_{x_1 \, kl}}
 \lr{ \lr{\phi_{x_1} G_{x_1 y_1} \ldots G_{x_n y_n}}_{ij} \,
 \expa{N \tr\ln\lr{D_{xy} + \phi_x \, \delta_{xy}} - \frac{N}{\lambda} \sum\limits_x \tr \phi_x } } = 0
\end{eqnarray}
Differentiating by parts and taking into account the last identity in (\ref{G_identities}), we arrive at the following relations:
\begin{eqnarray}
\label{sd_derivation2}
 \vev{G_{x_1 y_1} \ldots G_{x_n y_n}} - \vev{\phi_{x_1} G_{x_1 x_1}} \vev{G_{x_1 y_1} \ldots G_{x_n y_n}}
 - \nonumber \\ -
 \sum\limits_{A=2}^{n} \vev{\phi_{x_1} G_{x_1 y_1} \ldots G_{x_{A-1}y_{A-1}} G_{x_A x_1}} \vev{G_{x_1 y_A} G_{x_{A+1}y_{A+1}} \ldots G_{x_n y_n}}
 + \nonumber \\ +
 \vev{G_{x_1 x_1} \phi_{x_1} G_{x_1 y_1} \ldots G_{x_n y_n}}
 - \lambda^{-1} \vev{\phi_{x_1} G_{x_1 y_1} \ldots G_{x_n y_n} } = 0
\end{eqnarray}
In order to get the equations which involve only correlators of $G_{x y}$, we should now use the first two identities in (\ref{G_identities}) to get rid of $\phi_x$. However, there are several ways to do that which leads to different equations. Here we choose the way which does not produce the terms which can be further transformed by using the unitarity conditions. That is, in the third summand in (\ref{sd_derivation2}) we apply the second identity in (\ref{G_identities}) to $G_{x_A x_1}$, and in the fourth summand - to $G_{x_1 x_1} \xi_{x_1}$. As a result, we obtain the following generic set of equations:
\begin{eqnarray}
\label{sd_G_general}
 \vev{ G_{x_1 y_1} \ldots G_{x_n y_n} } \lr{1 + D_{x_1 z} \vev{G_{z x_1}}}
 +
 \lambda^{-1} D_{x_1 z} \vev{G_{z y_1} \ldots G_{x_n y_n}}
 -
 \lambda^{-1} \delta_{x_1 y_1} \vev{G_{x_2 y_2} \ldots G_{x_n y_n}}
 - \nonumber \\ -
 \sum\limits_{A=2}^{n} \delta_{x_1 x_A} \vev{G_{x_1 y_1} \ldots G_{x_{A-1} y_{A-1}}} \vev{G_{x_1 y_A} \ldots G_{x_n y_n}}
 + \nonumber \\ +
 \sum\limits_{A=2}^{n} \vev{G_{x_1 y_1} \ldots G_{x_{A-1} y_{A-1}} G_{x_A z}} D_{z x_1} \vev{G_{x_1 y_A} \ldots G_{x_n y_n}}
 - \nonumber \\ -
 D_{z x_1} \vev{G_{x_1 z} G_{x_1 y_1} \ldots G_{x_n y_n}} = 0 .
\end{eqnarray}
Let us first consider these equations in the coordinate space. It is then convenient to subtract the term proportional to the identity from $D_{xy}$ and write it as $D_{xy} = - \sum\limits_{\mu} \delta_{x, y \pm \hat{\mu}}$. The above equations can be then rewritten as
\begin{eqnarray}
\label{sd_G_coordinate_space}
 \vev{ G_{x_1 y_1} \ldots G_{x_n y_n} } \lr{1 + D_{x_1 z} \vev{G_{z x_1}}}
 +
 \lambda^{-1} D_{x_1 z} \vev{G_{z y_1} \ldots G_{x_n y_n}}
 -
 \lambda^{-1} \delta_{x_1 y_1} \vev{G_{x_2 y_2} \ldots G_{x_n y_n}}
 - \nonumber \\ -
 \sum\limits_{A=2}^{n} \delta_{x_1 x_A} \vev{G_{x_1 y_1} \ldots G_{x_{A-1} y_{A-1}}} \vev{G_{x_1 y_A} \ldots G_{x_n y_n}}
 + \nonumber \\ +
 \sum\limits_{A=2}^{n} \vev{G_{x_1 y_1} \ldots G_{x_{A-1} y_{A-1}} G_{x_A z}} D_{z x_1} \vev{G_{x_1 y_A} \ldots G_{x_n y_n}}
 - \nonumber \\ -
 D_{z x_1} \vev{G_{x_1 z} G_{x_1 y_1} \ldots G_{x_n y_n}} = 0 .
\end{eqnarray}


Introducing now the effective ``hopping parameter''
\begin{eqnarray}
\label{eff_hopping}
 \kappa = \lr{2 D \vev{g^{\dag}_x g_{x+\hat{\mu}}} - \lambda}
\end{eqnarray}
we can rewrite the above equations as
\begin{eqnarray}
\label{sd_G_general_kappa}
 \vev{ G_{x_1 y_1} \ldots G_{x_n y_n} }
 =
 -\kappa \delta_{x_1 y_1} \vev{G_{x_2 y_2} \ldots G_{x_n y_n}}
 -\kappa \sum\limits_{\mu} \vev{ G_{x_1 \pm \hat{\mu} y_1} \ldots G_{x_n y_n} }
 - \nonumber \\ -
 \lambda \kappa \sum\limits_{A=2}^{n} \delta_{x_1 x_A} \vev{G_{x_1 y_1} \ldots G_{x_{A-1} y_{A-1}}} \vev{G_{x_1 y_A} \ldots G_{x_n y_n}}
 - \nonumber \\ -
 \lambda \kappa \sum\limits_{A=2}^{n} \vev{G_{x_1 y_1} \ldots G_{x_{A} x_{1} \pm \hat{\mu}}} \vev{G_{x_1 y_A} \ldots G_{x_n y_n}}
 +
 \lambda \kappa \sum\limits_{\mu} \vev{G_{x_1 x_1 \pm \hat{\mu}} G_{x_1 y_1} \ldots G_{x_n y_n}}
\end{eqnarray}
Let us now go to the momentum space and define the new observables
\begin{eqnarray}
\label{momentum_space_observables}
 G_{p q} = \frac{1}{V^2} \sum\limits_{x, y} \expa{i p x + i q y} G_{x y},
\end{eqnarray}
where the momenta $p, q$ take discrete values $\vec{p} = 2 \pi \vec{m}/L$ and $L$ is the lattice size.
\begin{eqnarray}
\label{sd_G_general_kappa_momentum_preliminary}
 \vev{ G_{p_1 q_1} \ldots G_{p_n q_n} } \lr{m^2 + K^2\lr{p_1}}
 =
  \frac{1}{V} \delta\lr{p_1 + q_1} \vev{G_{p_2 q_2} \ldots G_{p_n q_n}}
 + \nonumber \\ +
 \frac{\lambda}{V} \sum\limits_{A=2}^{n} \delta\lr{p_1 + p_A - \tilde{p}_1 - \tilde{p}_A} \vev{G_{\tilde{p}_1 q_1} \ldots G_{p_{A-1} q_{A-1}}} \vev{G_{\tilde{p}_A q_A} \ldots G_{p_n q_n}}
 - \nonumber \\ -
 \lambda \sum\limits_{A=2}^{n}
 \sum\limits_{\tilde{p}_1 \tilde{q}_A \tilde{p}_A} K^2\lr{\tilde{q}_A}
 \delta\lr{p_1 - \tilde{p}_1 - \tilde{q}_A - \tilde{p}_A}
 \vev{G_{\tilde{p}_1 q_1} \ldots G_{p_{A} \tilde{q}_A}} \vev{G_{\tilde{p}_A q_A} \ldots G_{p_n q_n}}
 + \nonumber \\ +
 \lambda \sum\limits_{\tilde{p}_1, \tilde{p}_2, \tilde{q}_1}
 \delta\lr{p_1 - \tilde{p}_1 - \tilde{p}_2 - \tilde{q}_1} K^2\lr{\tilde{q}_1} \vev{G_{\tilde{p}_1 \tilde{q}_1} G_{\tilde{p}_2 q_1} G_{p_2 q_2} \ldots G_{p_n q_n}} ,
\end{eqnarray}
where $K^2\lr{p} = \sum\limits_{\mu} 4 \sin^2\lr{k_{\mu}/2}$ and $m^2 \equiv \lambda + 2 D\lr{1 - \vev{g^{\dag}_x g_{x + \hat{0}}}}$. Introducing now the ``propagator''
\begin{eqnarray}
\label{propagator_def}
 \mathcal{G}\lr{p} = \frac{1}{m^2 + K^2\lr{p}}
\end{eqnarray}
we can rewrite the above equations as
\begin{eqnarray}
\label{sd_G_general_kappa_momentum}
 \vev{ G_{p_1 q_1} \ldots G_{p_n q_n} }
 =
  \frac{1}{V} \mathcal{G}\lr{p_1} \delta\lr{p_1 + q_1} \vev{G_{p_2 q_2} \ldots G_{p_n q_n}}
 + \nonumber \\ +
 \frac{1}{V} \mathcal{G}\lr{p_1} \sum\limits_{A=2}^{n} \sum\limits_{\tilde{p}_1, \tilde{p}_A} \delta\lr{p_1 + p_A - \tilde{p}_1 - \tilde{p}_A} \vev{G_{\tilde{p}_1 q_1} \ldots G_{p_{A-1} q_{A-1}}} \vev{G_{\tilde{p}_A q_A} \ldots G_{p_n q_n}}
 - \nonumber \\ -
 \lambda \mathcal{G}\lr{p_1} \sum\limits_{A=2}^{n}
 \sum\limits_{\tilde{p}_1 \tilde{q}_A \tilde{p}_A}
 \delta\lr{p_1 - \tilde{p}_1 - \tilde{q}_A - \tilde{p}_A}
 K^2\lr{\tilde{q}_A}
 \vev{G_{\tilde{p}_1 q_1} \ldots G_{p_{A} \tilde{q}_A}} \vev{G_{\tilde{p}_A q_A} \ldots G_{p_n q_n}}
 + \nonumber \\ +
 \lambda \mathcal{G}\lr{p_1} \sum\limits_{\tilde{p}_1, \tilde{p}_2, \tilde{q}_1}
 \delta\lr{p_1 - \tilde{p}_1 - \tilde{p}_2 - \tilde{q}_1} K^2\lr{\tilde{q}_1} \vev{G_{\tilde{p}_1 \tilde{q}_1} G_{\tilde{p}_2 q_1} G_{p_2 q_2} \ldots G_{p_n q_n}}
\end{eqnarray}
\end{widetext}


\section*{$SU\lr{N}$ sigma-model: strong-coupling expansion in momentum space}
\label{sec:introduction}

 We study the theory defined by the integral over elements $g\lr{x}$ of $SU\lr{N}$, where $x$ labels different lattice sites. The partition function is:
\begin{eqnarray}
\label{pf_def}
\mathcal{Z} = \int\limits_{SU\lr{N}} \mathcal{D} g_x \,
\expa{-\frac{N}{\lambda}\, \sum \limits_{x,y} D_{xy} \tr g_x g^{\dag}_y } ,
\end{eqnarray}
where again we assume that $D_{xy}$ is the kinetic operator which grows towards larger momenta. From now on we will assume summations over repeated indices $x$, $y$ for the sake of brevity.

The single-trace observables which factorize in the large-$N$ limit are:
\begin{eqnarray}
\label{gf_def}
\mathcal{G}\lr{x_1, y_1, \ldots, x_n, y_n}
= \nonumber \\ =
\frac{1}{N}\, \vev{ \tr\lr{
g_{x_1} g^{\dag}_{y_1} \ldots g_{x_n} g^{\dag}_{y_n}} }
\end{eqnarray}
Let us now obtain the full set of Schwinger-Dyson equations for this theory, for example, by a variation over $g_{x_1}$:
\begin{widetext}
\begin{eqnarray}
\label{SD_derivation1}
\int\limits_{SU\lr{N}} \mathcal{D} g_{x} \, -i \nabla_a^{x_1} \,
\lr{g_{x_1} g^{\dag}_{y_1} \ldots g_{x_n} g^{\dag}_{y_n} \, \expa{ -\frac{N}{\lambda}\, \sum \limits_{x,y} D_{xy} \tr g_x g^{\dag}_y} } = 0
\end{eqnarray}
From this equation we obtain the following set of Schwinger-Dyson equations:
\begin{eqnarray}
\label{SDs_n2}
\mathcal{G}\lr{x_1, y_1} =
\delta_{x_1, y_1}
 - %\nonumber \\ -
\frac{1}{\lambda} D_{x_1 x} \, \mathcal{G}\lr{x, y_1}
 +
\frac{1}{\lambda} D_{x_1 x} \, \mathcal{G}\lr{x_1, x, x_1, y_1}
\end{eqnarray}

\begin{eqnarray}
\label{SDs}
\mathcal{G}\lr{x_1, y_1, \ldots, x_n, y_n}
= %\nonumber\\ =
\sum\limits_{A=2}^{n-1} \delta_{x_1, y_A} \,
\mathcal{G}\lr{x_A, y_1, \ldots, x_{A-1}, y_{A-1}}\,
\mathcal{G}\lr{x_{A+1}, y_{A+1}, \ldots, x_n, y_n}\,
+ \nonumber \\ +
\delta_{x_1, y_1} \, \mathcal{G}\lr{x_2, y_2, \ldots, x_n, y_n}\,
+
\delta_{x_1, y_n} \, \mathcal{G}\lr{x_n, y_1, \ldots, x_{n-1}, y_{n-1}}\,
- \nonumber \\ -
\sum\limits_{A=2}^{n} \delta_{x_1, x_A} \,
 \mathcal{G}\lr{x_1, y_1, \ldots, x_{A-1}, y_{A-1}}\,
 \mathcal{G}\lr{x_A, y_A, \ldots, x_n, y_n}\,
- \nonumber \\ -
\frac{1}{\lambda} D_{x_1 x} \mathcal{G}\lr{x, y_1, \ldots, x_n, y_n}
+
\frac{1}{\lambda} D_{x_1 x} \, \mathcal{G}\lr{x_1, x, x_1, y_1, \ldots, x_n, y_n}
\end{eqnarray}

Let us now define the correlators in the momentum space:
\begin{eqnarray}
\label{momentum_space_def}
\mathcal{G}\lr{p_1, q_1, \ldots, p_n, q_n} = \frac{1}{V^{2 n}} \,
\sum\limits_{x_1, y_1} \ldots \sum\limits_{x_n, y_n} \,
\nonumber \\
\expa{i \sum\limits_A p_A x_A + i \sum\limits_A q_A y_A} \,
\mathcal{G}\lr{x_1, y_1, \ldots, x_n, y_n}  ,
\end{eqnarray}
where $V$ is the total volume of space. In the momentum space the equations (\ref{SDs_n2}), (\ref{SDs}) read:
\begin{eqnarray}
\label{SDs_n2_momentum}
 \mathcal{G}\lr{p_1, q_1} = \lambda \, \frac{\mathcal{G}_0\lr{p_1} \delta\lr{p_1 + q_1}}{V}
 +
 \mathcal{G}_0\lr{p_1} \,
\sum\limits_{\tilde{p}_1, \tilde{q}_1, \tilde{p}_2}
\delta\lr{p_1, \tilde{p}_1 + \tilde{q}_1 + \tilde{p}_2} \,
D\lr{\tilde{q}_1} \mathcal{G}\lr{\tilde{p}_1, \tilde{q}_1, \tilde{p}_2, q_1}
\end{eqnarray}

\begin{eqnarray}
\label{SDs_pcm_momentum}
 \mathcal{G}\lr{p_1, q_1, \ldots, p_n, q_n}
 = \nonumber \\ =
 \sum\limits_{A=2}^{n-1}
 \lambda \, \frac{\mathcal{G}_0\lr{p_1} \, \delta\lr{p_1 + q_A}}{V} \,
 \mathcal{G}\lr{    p_A,     q_1, \ldots, p_{A-1}, q_{A-1}}
 \mathcal{G}\lr{p_{A+1}, q_{A+1}, \ldots,     p_n,    q_n }
 + \nonumber \\ +
 \lambda \, \frac{\mathcal{G}_0\lr{p_1} \, \delta\lr{p_1 + q_1}}{V} \,
 \mathcal{G}\lr{p_2, q_2, \ldots, p_n, q_n}
 + \nonumber \\ +
 \lambda \, \frac{\mathcal{G}_0\lr{p_1} \, \delta\lr{p_1 + q_n}}{V} \,
 \mathcal{G}\lr{p_n, q_1, p_2, q_2, \ldots, p_{n-1}, q_{n-1}}
 - \nonumber \\ -
 \lambda \, \frac{\mathcal{G}_0\lr{p_1}}{V} \,
 \sum\limits_{A=2}^{n}
 \sum\limits_{\tilde{p}_1 \tilde{p}_A} \delta\lr{p_1 + p_A, \tilde{p}_1 + \tilde{p}_A}
 \mathcal{G}\lr{\tilde{p}_1, q_1, p_2, q_2, \ldots, p_{A-1}, q_{A-1}}
 \mathcal{G}\lr{\tilde{p}_A, q_A,           \ldots, p_n, q_n}
 + \nonumber \\ +
 \mathcal{G}_0\lr{p_1}
 \sum\limits_{\tilde{p}_1, \tilde{q}_1, \tilde{p_2}}
 \delta\lr{p_1, \tilde{p}_1 + \tilde{q}_1 + \tilde{p}_2} D\lr{\tilde{q}_1}
 \mathcal{G}\lr{\tilde{p}_1, \tilde{q}_1, \tilde{p}_2, q_1, p_2, q_2, \ldots, p_n, q_n}
\end{eqnarray}
where we have defined the effective propagator
\begin{eqnarray}
\label{free_prop_def}
\mathcal{G}_0\lr{p} = \lr{\lambda + D\lr{p}}^{-1} .
\end{eqnarray}
\end{widetext}
For further convenience let us also define
\begin{eqnarray}
\label{prop_norm_def}
\Sigma = V^{-1} \sum\limits_{p} \mathcal{G}_0\lr{p}
\end{eqnarray}

Let us now, as usual, try to solve this equation by stochastic methods. We assume that $G\lr{p_1, q_1, \ldots, p_n, q_n} = \mathcal{N} c^{n} \, w\lr{p_1, q_1, \ldots, p_n, q_n}$, where $w\lr{p_1, q_1, \ldots, p_n, q_n}$ is the probability to encounter the sequence of momenta $\lrc{p_1, q_1, \ldots, p_n, q_n}$ in a certain random process, for which the equations (\ref{SDs_n2_pcm_momentum}) and (\ref{SDs_pcm_momentum}) are the steady-state equations. This random process is then given by the following set of random steps (actions):

\begin{description}
 \item[Create]: With probability $\frac{\Sigma}{\mathcal{N} c}$ create a new pair of momenta $\lrc{p, -p}$, where $p$ is distributed with probability distribution which is proportional to $G_0\lr{p}$.
 \item[Add momenta]: With probability $\frac{2 \Sigma}{c}$ add to the topmost sequence a pair of momenta $\lrc{p, -p}$ (distributed as for the previous action) either at the beginning of the sequence, or at the beginning and in the end: $\lrc{p_1, q_1, \ldots, p_n, q_n} \rightarrow \lrc{p, -p, p_1, q_1, \ldots, p_n, q_n}$ or $\lrc{p_1, q_1, \ldots, p_n, q_n} \rightarrow \lrc{p, q_1, \ldots, p_n, q_n, p_1, -p}$ (both choices are realized with equal probability).
 \item[Join sequences with random momenta]: With probability $\frac{\mathcal{N} \Sigma\lr{\lambda, \xi}}{c}$ take two sequences $\lrc{p_1, q_1, \ldots, p_m, q_m}$ and $\lrc{\tilde{p}_1, \tilde{q}_1, \ldots, \tilde{p}_m, \tilde{q}_m}$ from the stack and join them with a pair of random momenta $\lrc{p, -p}$ (also distributed as in the previous actions) as $\lrc{p, q_1, \ldots, p_m, q_m, p_1, -p, \tilde{p}_1, \tilde{q}_1, \ldots, \tilde{p}_m, \tilde{q}_m}$.
 \item[Flip momenta]: With probability $\mathcal{N} \Sigma\lr{\lambda, \xi}$ take two sequences $\lrc{p_1, q_1, \ldots, p_m, q_m}$ and $\lrc{\tilde{p}_1, \tilde{q}_1, \ldots, \tilde{p}_m, \tilde{q}_m}$ and combine them into the new sequence $\lrc{p_1', q_1, \ldots, p_m, q_m, \lr{p_1 + \tilde{p}_1 - p_1'}, \tilde{p}_1, \tilde{q}_1, \ldots, \tilde{p}_m, \tilde{q}_m}$, where $p_1'$ again has the probability distribution which is proportional to $G\lr{p_1'}$.
 \item[Create vertex]: With probability $G_0\lr{Q} \lr{||q_1||^2 + \xi}/\lambda$ take the sequence $\lrc{p_1, q_1, p_2, q_2, \ldots, p_n, q_n}$ and replace it with the sequence $\lrc{Q, q_2, p_3, q_3, \ldots, p_n, q_n}$, where $Q = p_1 + q_1 + p_2$.
\end{description}

\section{Some reference results for the Gross-Witten model}

 If the operator $D_{x y}$ in (\ref{pf_def}) is defined as $D_{x y} = 2 D \delta_{x,y} - \sum\limits_{\mu=1}^{D} \lr{\delta_{x+\hat{\mu},y} + \delta_{x-\hat{\mu},y}}$, then for the lattice of two sites the nontrivial terms in the action are $2/\lambda \tr\lr{g^{\dag}_0 g_1} + c.c.$, therefore $\lambda_{GW}^{-1} = 2/\lambda$ and $\lambda_{GW} = \lambda/2$. The strong-coupling correlator is
\begin{eqnarray}
\label{gw_sc_correlator}
 \mathcal{G}_{xy} = 
 \begin{cases}
                  1, & x = y \\
  \lambda_{GW}^{-1}, & x \neq y \\
 \end{cases}
\end{eqnarray}
The Fourier transform yields then $\mathcal{G}_{00} = \frac{1}{2} + \frac{1}{\lambda}$, $\mathcal{G}_{11} = \frac{1}{2} - \frac{1}{\lambda}$.


\begin{acknowledgments}
 This work was supported by the S.~Kowalevskaja award from the Alexander von Humboldt foundation.
\end{acknowledgments}

\bibliography{MyBibliography}
\bibliographystyle{apsrev}

\end{document}
