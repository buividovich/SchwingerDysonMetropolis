\documentclass[twocolumn,showpacs,preprintnumbers,superscriptaddress,amsmath,floatfix,amssymb,secnumarabic]{revtex4}
\usepackage[colorlinks=true]{hyperref}
%\usepackage[colorlinks=false]{hyperref}
\usepackage{graphicx}

\newcommand{\CG}{\mathfrak{G}}
\newcommand{\G}{\mathcal{G}}

\newcommand{\comment}[1]{}

\newcommand{\lr}[1]{ \left( #1 \right) }
\newcommand{\lrs}[1]{ \left[ #1 \right] }
\newcommand{\lrc}[1]{ \left\{ #1 \right\} }
\newcommand{\vev}[1]{ \langle \, #1 \, \rangle }

\newcommand{\Tr}{ {\rm Tr} \, }
\newcommand{\tr}{ {\rm Tr} \, }
\newcommand{\re}{ {\rm Re} \, }
\newcommand{\im}{ {\rm Im} \, }
\renewcommand{\Re}{ {\rm Re} \, }
\renewcommand{\Im}{ {\rm Im} \, }

\newcommand{\rvac}{ \, | 0 \rangle }
\newcommand{\lvac}{ \langle 0 | \, }
\newcommand{\ket}[1]{ \, | #1 \rangle }
\newcommand{\bra}[1]{ \langle #1 | \, }

\newcommand{\diag}[1]{ {\rm diag} \, \left( #1 \right) }
\newcommand{\const}{ {\rm const}}
\renewcommand{\det}[1]{ {\rm det} \left( #1 \right) }

\newcommand{\sign}{ {\rm sign} \,  }
\newcommand{\sh}{ {\rm sh} \,  }
\newcommand{\ch}{ {\rm ch} \,  }
\renewcommand{\th}{ {\rm th} \,  }
\newcommand{\hodge}{{}^{*}}
\newcommand{\expa}[1]{ \exp{\left( #1 \right)} }
\newcommand{\abs}[1]{| #1 |}
\newcommand{\zvar}[1]{ e^{\frac{2 \pi i}{N} \, #1  } }
\newcommand{\zvarbr}[1]{ \exp{ \left( \frac{2 \pi i}{N} \left( #1 \right) \right) }}

\newcommand{\red}[1]{\textcolor[rgb]{1.00,0.00,0.00}{#1}}

\newcommand{\logo}{\\ \vskip -18mm
\leftline{\includegraphics[scale=0.3,clip=false]{logo.eps}} \vskip 10mm}

% PACS: 02.70.-c Computational techniques; simulations
%       02.50.Ey Stochastic processes
%       11.15.Pg Expansions for large numbers of components e.g., 1/Nc expansions


\begin{document}
\sloppy

\title{Numerical solution of Schwinger-Dyson equations in large-$N$ $U\lr{N}$ sigma model}

\author{P. V. Buividovich}
\email{pavel.buividovich@physik.uni-regensburg.de}
\affiliation{Regensburg University, Regensburg, Germany}

\date{June 19, 2014}
\begin{abstract}
 We propose a numerical stochastic algorithm for solving the Schwinger-Dyson equations in large-$N$ $U\lr{N}$ sigma model.
\end{abstract}
\pacs{02.70.-c; 02.50.Ey; 11.15.Pg}

\maketitle

\section*{Definitions}

For further convenience, let us introduce the notation
\begin{eqnarray}
\label{vev_definition}
 \vev{F\lr{\phi}} = \mathcal{Z}^{-1} \int\limits_{\mathbb{C}^{N \times N}} \mathcal{D}\phi e^{-S\lrs{\Phi}} \frac{1}{N} \tr F\lr{\phi}
\end{eqnarray}
for the expectation value of the normalized trace of some matrix-valued observable $F\lrs{\phi}$, where $\phi$ is some $N \times N$ matrix-valued field.

\section*{Unitarity via (implicit) matrix Lagrange multiplier}

 We start with the partition function of the form
\begin{eqnarray}
\label{sm_partition}
 \mathcal{Z} = \int\limits_{U\lr{N}} dg_x \expa{-\frac{N}{\lambda} \sum\limits_{x,y} D_{xy} \tr\lr{g^{\dag}_x g_y} } ,
\end{eqnarray}
where $D_{xy} = 2 D \delta_{x y} - \sum\limits_{\mu} \delta_{x,y+\hat{\mu}} - \sum\limits_{\mu} \delta_{x,y-\hat{\mu}}$ is the lattice discretization of the continuum operator $-\partial_{\mu}^2$. It is important that this operator is positive-semidefinite. Its eigenvalues are $\lambda \sim \sum\limits_{\mu} 4 \sin^2\lr{k_{\mu}/2}$.

Let us now replace integration over unitary matrices $g_x$ by integration over the full $\mathbb{C}^{N \times N}$ and enforce the unitarity condition $g^{\dag}_x g_x$ by an integral over a hermitian-matrix-valued Lagrange multiplier $\phi_x$:
\begin{widetext}
\begin{eqnarray}
\label{sm_partition_matrix_lagrange}
 \mathcal{Z} = \int dg_x \int d\phi_x \expa{ - \frac{N}{\lambda} \sum\limits_{x \neq y} D_{xy} \tr\lr{g^{\dag}_x g_y} - i \frac{N}{\lambda} \sum\limits_x \tr\lr{\phi_x  g^{\dag}_x g_x - \phi_x} }
\end{eqnarray}
Performing the Gaussian integral over $g_x$ and $g^{\dag}_x$, we arrive at the following representation:
\begin{eqnarray}
\label{sm_partition_matrix_lagrange_only}
 \mathcal{Z} = \int d\phi_x
 \expa{- N \tr\ln\lr{D_{xy} + i \phi_x \delta_{xy}} + i \frac{N}{\lambda} \sum\limits_x \tr \phi_x } .
\end{eqnarray}
Note that as such, this integral is not well defined over the manifold of Hermitian matrices. However, one can still write the SD equations which will pick up the stable thimble going through the saddle point with positive eigenvalues of $\phi_x$. At the naive saddle point of (\ref{sm_partition_matrix_lagrange_only}), where $\phi_x$ is proportional to the identity, the mean-field value of $\phi$ scales as $e^{-4 \pi/\lambda}$, whereas the relevant mass parameter scales as $m^2 \sim e^{-\frac{16 \pi}{\lambda}}$.
\end{widetext}

 The observables like $\vev{g^{\dag}_x g_y }$ are now expressed in terms of $\vev{G_{xy}}$ where $G_{xy}$ is the inverse of the operator $D_{xy} + \phi_x \delta_{xy}$ (which acts on the space of $N$-component fields $\psi_{x i}$). In particular, $\vev{g^{\dag}_x g_y } = \lambda \, \vev{G_{xy}}$. By definition, $G_{xy}$ satisfy the identities of the form
\begin{eqnarray}
\label{G_identities}
 \sum\limits_z D_{xz} G_{zy} + i \phi_x G_{xy} = \delta_{x y} ,
 \nonumber \\
 \sum\limits_z G_{xz} D_{zy} + G_{xy} i \phi_y = \delta_{x y} .
\end{eqnarray}
Note that these are now the operator identities which hold on configuration per configuration basis.

 Schwinger-Dyson equations for the path integral (\ref{sm_partition_matrix_lagrange_only}) can be derived in terms of several sets of observables. One could, for example, consider the correlators of the $\phi_x$ variable itself. This choice, however, leads to many nonlocal terms in the action. A local form of SD equations can be obtained in terms of the observables of the form $\vev{G_{x_1 y_1} \ldots G_{x_n y_n}}$.

 In order to derive the Schwinger-Dyson equations for the above observables, let us consider the following full derivative:
\begin{widetext}
\begin{eqnarray}
\label{sd_derivation1}
 \delta_{ik} \delta_{jl} \, \int\mathcal{D}\phi_x \frac{\partial}{\partial \phi_{x_1 \, kl}}
 \lr{ \lr{\phi_{x_1} G_{x_1 y_1} \ldots G_{x_n y_n}}_{ij} \,
 \expa{- N \tr\ln\lr{D_{xy} + i \phi_x \, \delta_{xy}} + i \frac{N}{\lambda} \sum\limits_x \tr \phi_x } } = 0
\end{eqnarray}
Differentiating by parts and taking into account the last identity in (\ref{G_identities}), we arrive at the following relations:
\begin{eqnarray}
\label{sd_derivation2_lowest}
 \vev{G_{x y}} - \vev{i \phi_x G_{x x}} \vev{G_{x y}}
 - \vev{G_{x x} i \phi_x G_{x y} } + \lambda^{-1} \vev{i \phi_x G_{x y}} = 0\\{}\nonumber\\
\label{sd_derivation2}
 \vev{G_{x_1 y_1} \ldots G_{x_n y_n}} - \vev{i \phi_{x_1} G_{x_1 x_1}} \vev{G_{x_1 y_1} \ldots G_{x_n y_n}}
 - \nonumber \\ -
 \sum\limits_{A=2}^{n} \vev{i \phi_{x_1} G_{x_1 y_1} \ldots G_{x_{A-1}y_{A-1}} G_{x_A x_1}} \vev{G_{x_1 y_A} G_{x_{A+1}y_{A+1}} \ldots G_{x_n y_n}}
 - \nonumber \\ -
 \vev{G_{x_1 x_1} i \phi_{x_1} G_{x_1 y_1} \ldots G_{x_n y_n}}
 + \lambda^{-1} \vev{i \phi_{x_1} G_{x_1 y_1} \ldots G_{x_n y_n} } = 0
\end{eqnarray}
In order to get the equations which involve only correlators of $G_{x y}$, we should now use the first two identities in (\ref{G_identities}) to get rid of $\phi_x$. However, there are several ways to do that which leads to different equations. Here we choose the way which does not produce the terms which can be further transformed by using the unitarity conditions. That is, the summands of the form $\vev{i \phi_{x_1} G_{x_1 y_1} \ldots G_{x_{A-1}y_{A-1}} G_{x_A x_1}}$ are replaced by
\begin{eqnarray}
\label{phix_replacement1}
 \vev{ G_{x_1 y_1} \ldots G_{x_{A-1}y_{A-1}} G_{x_A x_1} i \phi_{x_1}} =
 \delta_{x_A x_1} \, \vev{ G_{x_1 y_1} \ldots G_{x_{A-1}y_{A-1}}} - D_{z x_1} \vev{ G_{x_1 y_1} \ldots G_{x_{A-1}y_{A-1}} G_{x_A z} } ,
\end{eqnarray}
and the summands of the form $\vev{G_{x_1 x_1} i \phi_{x_1} G_{x_1 y_1} \ldots G_{x_n y_n}}$ - by
\begin{eqnarray}
\label{phix_replacement2}
 \vev{G_{x_1 x_1} i \phi_{x_1} G_{x_1 y_1} \ldots G_{x_n y_n}} = \vev{G_{x_1 y_1} \ldots G_{x_n y_n}} - D_{z x_1} \vev{G_{x_1 z} G_{x_1 y_1} \ldots G_{x_n y_n}} ,
\end{eqnarray}
where we assume summation over the index $z$. As a result, we obtain the following set of equations:
\begin{eqnarray}
\label{sd_G_general_lowest}
 \lr{\lambda \vev{i \phi_t G_{tt}} \delta_{x z} + D_{x z}} \vev{G_{z y}}
 = \delta_{x y} + \lambda D_{z x} \vev{G_{x z} G_{x y}}
 \\{}\nonumber\\
 \label{sd_G_general}
 \lr{\lambda \vev{i \phi_t G_{tt}} \delta_{x_1 z} + D_{x_1 z}} \vev{G_{z y_1} G_{x_2 y_2} \ldots G_{x_n y_n}}
 = \delta_{x_1 y_1} \vev{G_{x_2 y_2} \ldots G_{x_n y_n}}
 - \nonumber \\ -
 \lambda \sum\limits_{A=2}^{n} \delta_{x_1 x_A} \vev{G_{x_1 y_1} \ldots G_{x_{A-1} y_{A-1}}} \vev{G_{x_1 y_A} \ldots G_{x_n y_n}}
 + \nonumber \\ +
 \lambda \sum\limits_{A=2}^{n} D_{x_1 z} \vev{G_{x_1 y_1} \ldots G_{x_{A-1} y_{A-1}} G_{x_A z}} \vev{G_{x_1 y_A} \ldots G_{x_n y_n}}
 + \nonumber \\ +
 \lambda D_{z x_1} \vev{G_{x_1 z} G_{x_1 y_1} \ldots G_{x_n y_n}} = 0 ,
\end{eqnarray}
where we have also used the translational invariance to denote $\vev{i \phi_{x_1} G_{x_1 x_1}} = \vev{i \phi_t G_{tt}}$, where $t$ is an arbitrary point on the lattice. From the structure of these equations we see that the bare propagators acquire the effective mass term
\begin{eqnarray}
\label{eff_mass_lm}
 m^2 = \lambda \vev{i \phi_t G_{tt}} = \lambda - \lambda G_{t z} D_{z t}
 = \lambda - 2 D \lr{1 - \vev{g^{\dag}_t g_{t + \hat{0}}} }
\end{eqnarray}
which self-consistently depends on the mean link expectation value $\vev{g^{\dag}_t g_{t + \hat{0}}}$ (the above formula is ONLY valid for isotropic lattice - at finite temperature there will be the average over different directions).

Let us now go to the momentum space and define the new observables
\begin{eqnarray}
\label{momentum_space_observables}
 G_{p q} = \frac{1}{V^2} \sum\limits_{x, y} \expa{i p x + i q y} G_{x y},
\end{eqnarray}
where the momenta $p, q$ take discrete values $\vec{p} = 2 \pi \vec{m}/L$, $L$ is the lattice size and $V = L^D$ is the lattice volume. In terms of the new observables the equations (\ref{sd_G_general_lowest}) and (\ref{sd_G_general}) take the following form:
\begin{eqnarray}
\label{sd_G_general_fourier_lowest}
 \vev{G_{p q}} = \frac{\G\lr{p} \delta_{p+q}}{V} +
 \lambda \G\lr{p} \sum\limits_{\tilde{p}_1, \tilde{p}_2, \tilde{q}_1}
 \delta\lr{p - \tilde{p}_1 - \tilde{p}_2 - \tilde{q}_1} D\lr{\tilde{q}_1} \vev{G_{\tilde{p}_1 \tilde{q}_1} G_{\tilde{p}_2 q}}
 \\{}\nonumber\\
\label{sd_G_general_fourier}
 \vev{ G_{p_1 q_1} \ldots G_{p_n q_n} }
 =
  \frac{\G\lr{p_1} \delta_{p_1 + q_1}}{V} \, \vev{G_{p_2 q_2} \ldots G_{p_n q_n}}
 - \nonumber \\ -
 \frac{\lambda}{V} \G\lr{p_1} \sum\limits_{A=2}^{n} \sum\limits_{\tilde{p}_1, \tilde{p}_A} \delta\lr{p_1 + p_A - \tilde{p}_1 - \tilde{p}_A} \vev{G_{\tilde{p}_1 q_1} \ldots G_{p_{A-1} q_{A-1}}} \vev{G_{\tilde{p}_A q_A} \ldots G_{p_n q_n}}
 + \nonumber \\ +
 \lambda \G\lr{p_1} \sum\limits_{A=2}^{n}
 \sum\limits_{\tilde{p}_1 \tilde{q}_A \tilde{p}_A}
 \delta\lr{p_1 - \tilde{p}_1 - \tilde{q}_A - \tilde{p}_A}
 D\lr{\tilde{q}_A}
 \vev{G_{\tilde{p}_1 q_1} \ldots G_{p_{A} \tilde{q}_A}} \vev{G_{\tilde{p}_A q_A} \ldots G_{p_n q_n}}
 + \nonumber \\ +
 \lambda \G\lr{p_1} \sum\limits_{\tilde{p}_1, \tilde{p}_2, \tilde{q}_1}
 \delta\lr{p_1 - \tilde{p}_1 - \tilde{p}_2 - \tilde{q}_1} D\lr{\tilde{q}_1} \vev{G_{\tilde{p}_1 \tilde{q}_1} G_{\tilde{p}_2 q_1} G_{p_2 q_2} \ldots G_{p_n q_n}} ,
\end{eqnarray}
where we have introduced the propagator
\begin{eqnarray}
 \G\lr{p} = \frac{1}{m^2 + D\lr{p}},
 \quad
 D\lr{p} = \sum\limits_{\mu} 4 \sin^2\lr{\frac{k_{\mu}}{2}} .
\end{eqnarray}
\end{widetext}

\section*{Lowest-order perturbative expansion}

\begin{widetext}
\begin{eqnarray}
\label{G2_1}
 G^{\lr{1}}_{pq} = \frac{\G\lr{p} \G\lr{q} \delta_{p+q}}{V} \, I_0 ,
 \quad
 I_0 = \frac{1}{V} \sum\limits_p D\lr{p} \G\lr{p}
\end{eqnarray}
\begin{eqnarray}
\label{G4_1}
 G^{\lr{1}}_{p_1 q_1 p_2 q_2}
 =
 I_0 \,
 \frac{\delta_{p_1 + q_1}}{V}
 \frac{\delta_{p_2 + q_2}}{V} \,
 \lr{
  \G\lr{p_1} \G\lr{p_2} \G\lr{q_2}
  +
  \G\lr{p_1} \G\lr{q_1} \G\lr{p_2}
 }
 - \nonumber \\ -
 m^2 \,
 \frac{\delta_{p_1 + q_1 + p_2 + q_2}}{V^3} \,
 \G\lr{p_1} \G\lr{q_1} \G\lr{p_2} \G\lr{q_2}
\end{eqnarray}
Note that the last term which is completely symmetric w.r.t. permutations of all momenta appears
due to the cancellation of the terms $-\G\lr{p_1} \G\lr{q_1} \G\lr{q_2}$ and $\G\lr{p_1} \G\lr{q_1} \G\lr{p_2} \G\lr{q_2} D\lr{p_2}$.

\begin{eqnarray}
\label{G2_2}
 G^{\lr{2}}_{pq}
 =
 \G\lr{p}
 \sum\limits_{\tilde{p}_1, \tilde{q}_1, \tilde{p}_2 }
 \delta_{p - \tilde{p}_1 - \tilde{q}_1 - \tilde{p}_2}
 D\lr{\tilde{q}_1}
 G^{\lr{1}}_{\tilde{p}_1 \tilde{q}_1 \tilde{p}_2 q}
 = \nonumber \\ =
 \frac{\delta_{p+q}}{V} I_0^2 \G^3\lr{p}
 +
 \frac{\delta_{p+q}}{V} \G^2\lr{p} \lr{I_0 I_1 - m^2 S_0^2}
 +
 \frac{m^4}{V^3} \G^2\lr{p} \delta_{p+q}
 \sum\limits_{\tilde{p}, \tilde{q}}
 \G\lr{p - \tilde{p} - \tilde{q}}
 \G\lr{\tilde{p}}
 \G\lr{\tilde{q}} ,
\end{eqnarray}
where we have defined
\begin{eqnarray}
\label{scalar_ints_def}
 I_1 = \frac{1}{V} \sum\limits_q D\lr{q} \G^2\lr{q},
 \quad
 S_0 = \frac{1}{V} \sum\limits_q \G\lr{q} .
\end{eqnarray}

The first, the second and the third summands clearly correspond to the two tadpoles, tadpole on tadpole diagram and the sunset diagram, respectively.
\end{widetext}

\begin{figure*}
  \centering
  \includegraphics[width=6cm,angle=-90]{{../plots/smlm/G2_total_vs_ao_l0.60_morestat.eps}}
  \includegraphics[width=6cm,angle=-90]{{../plots/smlm/G2_total_vs_ao_l0.60.eps}}\\
  \includegraphics[width=6cm,angle=-90]{{../plots/smlm/G2_total_vs_ao_l1.20.eps}}
  \includegraphics[width=6cm,angle=-90]{{../plots/smlm/G2_total_vs_ao_l2.30.eps}}\\
  \caption{Contributions from the resummed expansion: $\lambda = 1.2$ (left) and $\lambda = 2.3$ (right)}
  \label{fig:contribs}
\end{figure*}

 For the GWW model, we see that the next-to-leading order of the formal iteration of the above equations already gives quite a good estimate, but then the expansion starts to diverge. We can check what happens with the PCM using the numerical data from \cite{Rossi:94:1}. Unfortunately, it seems that our expansion leads to a certain "overshooting" - the corrections are too large because of the too strong IR behavior.

\section*{$SU\lr{N}$ sigma-model: strong-coupling expansion in momentum space}
\label{sec:introduction}

 We study the theory defined by the integral over elements $g\lr{x}$ of $SU\lr{N}$, where $x$ labels different lattice sites. The partition function is:
\begin{eqnarray}
\label{pf_def}
\mathcal{Z} = \int\limits_{SU\lr{N}} \mathcal{D} g_x \,
\expa{-\frac{N}{\lambda}\, \sum \limits_{x,y} D_{xy} \tr g_x g^{\dag}_y } ,
\end{eqnarray}
where again we assume that $D_{xy}$ is the kinetic operator which grows towards larger momenta. From now on we will assume summations over repeated indices $x$, $y$ for the sake of brevity.

The single-trace observables which factorize in the large-$N$ limit are:
\begin{eqnarray}
\label{gf_def}
\G\lr{x_1, y_1, \ldots, x_n, y_n}
= \nonumber \\ =
\frac{1}{N}\, \vev{ \tr\lr{
g_{x_1} g^{\dag}_{y_1} \ldots g_{x_n} g^{\dag}_{y_n}} }
\end{eqnarray}
Let us now obtain the full set of Schwinger-Dyson equations for this theory, for example, by a variation over $g_{x_1}$ (these equations were first derived in \cite{GonzalezArroyo:84:1}):
\begin{widetext}
\begin{eqnarray}
\label{SD_derivation1}
\int\limits_{SU\lr{N}} \mathcal{D} g_{x} \, -i \nabla_a^{x_1} \,
\lr{g_{x_1} g^{\dag}_{y_1} \ldots g_{x_n} g^{\dag}_{y_n} \, \expa{ -\frac{N}{\lambda}\, \sum \limits_{x,y} D_{xy} \tr g_x g^{\dag}_y} } = 0
\end{eqnarray}
From this equation we obtain the following set of Schwinger-Dyson equations:
\begin{eqnarray}
\label{SDs_n2}
\G\lr{x_1, y_1} =
\delta_{x_1, y_1}
 - %\nonumber \\ -
\frac{1}{\lambda} D_{x_1 x} \, \G\lr{x, y_1}
 +
\frac{1}{\lambda} D_{x_1 x} \, \G\lr{x_1, x, x_1, y_1}
\end{eqnarray}

\begin{eqnarray}
\label{SDs}
\G\lr{x_1, y_1, \ldots, x_n, y_n}
= %\nonumber\\ =
\sum\limits_{A=2}^{n-1} \delta_{x_1, y_A} \,
\G\lr{x_A, y_1, \ldots, x_{A-1}, y_{A-1}}\,
\G\lr{x_{A+1}, y_{A+1}, \ldots, x_n, y_n}\,
+ \nonumber \\ +
\delta_{x_1, y_1} \, \G\lr{x_2, y_2, \ldots, x_n, y_n}\,
+
\delta_{x_1, y_n} \, \G\lr{x_n, y_1, \ldots, x_{n-1}, y_{n-1}}\,
- \nonumber \\ -
\sum\limits_{A=2}^{n} \delta_{x_1, x_A} \,
 \G\lr{x_1, y_1, \ldots, x_{A-1}, y_{A-1}}\,
 \G\lr{x_A, y_A, \ldots, x_n, y_n}\,
- \nonumber \\ -
\frac{1}{\lambda} D_{x_1 x} \G\lr{x, y_1, \ldots, x_n, y_n}
+
\frac{1}{\lambda} D_{x_1 x} \, \G\lr{x_1, x, x_1, y_1, \ldots, x_n, y_n}
\end{eqnarray}

Let us now define the correlators in the momentum space:
\begin{eqnarray}
\label{momentum_space_def}
\G\lr{p_1, q_1, \ldots, p_n, q_n} = \frac{\lambda^{-n}}{V^{2 n}} \,
\sum\limits_{x_1, y_1} \ldots \sum\limits_{x_n, y_n} \,
\nonumber \\
\expa{i \sum\limits_A p_A x_A + i \sum\limits_A q_A y_A} \,
\G\lr{x_1, y_1, \ldots, x_n, y_n}  ,
\end{eqnarray}
where $V$ is the total volume of space. In the momentum space the equations (\ref{SDs_n2}), (\ref{SDs}) read:
\begin{eqnarray}
\label{SDs_n2_momentum}
 \G\lr{p_1, q_1} = \frac{\G_0\lr{p_1} \delta\lr{p_1 + q_1}}{V}
 +
 \lambda \G_0\lr{p_1} \,
\sum\limits_{\tilde{p}_1, \tilde{q}_1, \tilde{p}_2}
\delta\lr{p_1, \tilde{p}_1 + \tilde{q}_1 + \tilde{p}_2} \,
D\lr{\tilde{q}_1} \G\lr{\tilde{p}_1, \tilde{q}_1, \tilde{p}_2, q_1}
\end{eqnarray}

\begin{eqnarray}
\label{SDs_pcm_momentum}
 \G\lr{p_1, q_1, \ldots, p_n, q_n}
 = \nonumber \\ =
 \sum\limits_{A=2}^{n-1}
 \frac{\G_0\lr{p_1} \, \delta\lr{p_1 + q_A}}{V} \,
 \G\lr{    p_A,     q_1, \ldots, p_{A-1}, q_{A-1}}
 \G\lr{p_{A+1}, q_{A+1}, \ldots,     p_n,    q_n }
 + \nonumber \\ +
 \frac{\G_0\lr{p_1} \, \delta\lr{p_1 + q_1}}{V} \,
 \G\lr{p_2, q_2, \ldots, p_n, q_n}
 + \nonumber \\ +
 \frac{\G_0\lr{p_1} \, \delta\lr{p_1 + q_n}}{V} \,
 \G\lr{p_n, q_1, p_2, q_2, \ldots, p_{n-1}, q_{n-1}}
 - \nonumber \\ -
 \lambda \, \frac{\G_0\lr{p_1}}{V} \,
 \sum\limits_{A=2}^{n}
 \sum\limits_{\tilde{p}_1 \tilde{p}_A} \delta\lr{p_1 + p_A, \tilde{p}_1 + \tilde{p}_A}
 \G\lr{\tilde{p}_1, q_1, p_2, q_2, \ldots, p_{A-1}, q_{A-1}}
 \G\lr{\tilde{p}_A, q_A,           \ldots, p_n, q_n}
 + \nonumber \\ +
 \lambda \G_0\lr{p_1}
 \sum\limits_{\tilde{p}_1, \tilde{q}_1, \tilde{p_2}}
 \delta\lr{p_1, \tilde{p}_1 + \tilde{q}_1 + \tilde{p}_2} D\lr{\tilde{q}_1}
 \G\lr{\tilde{p}_1, \tilde{q}_1, \tilde{p}_2, q_1, p_2, q_2, \ldots, p_n, q_n}
\end{eqnarray}
where we have defined the effective propagator
\begin{eqnarray}
\label{free_prop_def}
\G_0\lr{p} = \lr{\lambda + D\lr{p}}^{-1} .
\end{eqnarray}
\end{widetext}
For further convenience let us also define
\begin{eqnarray}
\label{prop_norm_def}
\Sigma = V^{-1} \sum\limits_{p} \G_0\lr{p}
\end{eqnarray}

Let us now, as usual, try to solve this equation by stochastic methods. We assume that $G\lr{p_1, q_1, \ldots, p_n, q_n} = \mathcal{N} c^{n} \, w\lr{p_1, q_1, \ldots, p_n, q_n}$, where $w\lr{p_1, q_1, \ldots, p_n, q_n}$ is the probability to encounter the sequence of momenta $\lrc{p_1, q_1, \ldots, p_n, q_n}$ in a certain random process, for which the equations (\ref{SDs_n2_pcm_momentum}) and (\ref{SDs_pcm_momentum}) are the steady-state equations. This random process is then given by the following set of random steps (actions):

\begin{description}
 \item[Create]: With probability $\frac{\Sigma}{\mathcal{N} c}$ create a new pair of momenta $\lrc{p, -p}$, where $p$ is distributed with probability distribution which is proportional to $G_0\lr{p}$.
 \item[Add momenta]: With probability $\frac{2 \Sigma}{c}$ add to the topmost sequence a pair of momenta $\lrc{p, -p}$ (distributed as for the previous action) either at the beginning of the sequence, or at the beginning and in the end: $\lrc{p_1, q_1, \ldots, p_n, q_n} \rightarrow \lrc{p, -p, p_1, q_1, \ldots, p_n, q_n}$ or $\lrc{p_1, q_1, \ldots, p_n, q_n} \rightarrow \lrc{p, q_1, \ldots, p_n, q_n, p_1, -p}$ (both choices are realized with equal probability).
 \item[Join sequences with random momenta]: With probability $\frac{\mathcal{N} \Sigma\lr{\lambda, \xi}}{c}$ take two sequences $\lrc{p_1, q_1, \ldots, p_m, q_m}$ and $\lrc{\tilde{p}_1, \tilde{q}_1, \ldots, \tilde{p}_m, \tilde{q}_m}$ from the stack and join them with a pair of random momenta $\lrc{p, -p}$ (also distributed as in the previous actions) as $\lrc{p, q_1, \ldots, p_m, q_m, p_1, -p, \tilde{p}_1, \tilde{q}_1, \ldots, \tilde{p}_m, \tilde{q}_m}$.
 \item[Flip momenta]: With probability $\mathcal{N} \Sigma\lr{\lambda, \xi}$ take two sequences $\lrc{p_1, q_1, \ldots, p_m, q_m}$ and $\lrc{\tilde{p}_1, \tilde{q}_1, \ldots, \tilde{p}_m, \tilde{q}_m}$ and combine them into the new sequence $\lrc{p_1', q_1, \ldots, p_m, q_m, \lr{p_1 + \tilde{p}_1 - p_1'}, \tilde{p}_1, \tilde{q}_1, \ldots, \tilde{p}_m, \tilde{q}_m}$, where $p_1'$ again has the probability distribution which is proportional to $G\lr{p_1'}$.
 \item[Create vertex]: With probability $G_0\lr{Q} \lr{||q_1||^2 + \xi}/\lambda$ take the sequence $\lrc{p_1, q_1, p_2, q_2, \ldots, p_n, q_n}$ and replace it with the sequence $\lrc{Q, q_2, p_3, q_3, \ldots, p_n, q_n}$, where $Q = p_1 + q_1 + p_2$.
\end{description}

\section{Some reference results for the Gross-Witten model}

 If the operator $D_{x y}$ in (\ref{pf_def}) is defined as $D_{x y} = 2 D \delta_{x,y} - \sum\limits_{\mu=1}^{D} \lr{\delta_{x+\hat{\mu},y} + \delta_{x-\hat{\mu},y}}$, then for the lattice of two sites the nontrivial terms in the action are $2/\lambda \tr\lr{g^{\dag}_0 g_1} + c.c.$, therefore $\lambda_{GW}^{-1} = 2/\lambda$ and $\lambda_{GW} = \lambda/2$. The strong-coupling correlator is
\begin{eqnarray}
\label{gw_sc_correlator}
 \G_{xy} =
 \begin{cases}
                  1, & x = y \\
  \lambda_{GW}^{-1}, & x \neq y \\
 \end{cases}
\end{eqnarray}
The Fourier transform yields then $\G_{00} = \frac{1}{2} + \frac{1}{\lambda}$, $\G_{11} = \frac{1}{2} - \frac{1}{\lambda}$.

In the weak-coupling limit, we have
\begin{eqnarray}
\label{gw_sc_correlator}
 \G_{xy} =
 \begin{cases}
                   1, & x = y \\
  1 - \lambda_{GW}/4, & x \neq y \\
 \end{cases} ,
\end{eqnarray}
which yields after the Fourier transform $\G_{00} = 1 - \lambda_{GW}/8 = 1 - \lambda/16$, $\G_{11} = \lambda_{GW}/8 = \lambda/16$.

$SU\lr{N=\infty}$ free algebra:
\begin{eqnarray}
\label{sun_free_algebra}
 \G^{0}_{x_1 y_1} = \int\mathcal{D}g \frac{1}{N} \tr\lr{g_{x_1} g^{\dag}_{y_1}}
 = \delta_{x_1 y_1}
\nonumber \\
 \G^{0}_{x_1 y_1 x_2 y_2} = \int\mathcal{D}g \frac{1}{N} \tr\lr{g_{x_1} g^{\dag}_{y_1} g_{x_2} g^{\dag}_{y_2}}
 = \nonumber \\ =
 \delta_{x_1 y_1} \delta_{x_2 y_2} + \delta_{x_1 y_2} \delta_{y_1 x_2} - \delta_{x_1 x_2} \delta_{x_1 y_1} \delta_{x_2 y_2}
\end{eqnarray}

In momentum space these free correlators read
\begin{eqnarray}
\label{sun_free_algebra_mspace}
 \G^{0}_{p_1 q_1} = \frac{1}{V} \delta\lr{p_1 + q_1}
\nonumber \\
 \G^{0}_{p_1 q_1 p_2 q_2} =
 \frac{1}{V^2} \delta\lr{p_1 + q_1} \delta\lr{p_2 + q_2}
 + \nonumber \\ +
 \frac{1}{V^2} \delta\lr{p_1 + q_2} \delta\lr{p_2 + q_1}
 - \nonumber \\ -
 \frac{1}{V^3} \delta\lr{p_1 + q_1 + p_2 + q_2}
\end{eqnarray}

\section{Automated tuning of $c$ and $\mathcal{N}$ coefficients}

 For the stochastic solution of a linear system of the form $\ket{\phi} = A \ket{\phi} + \ket{b}$, the expectation value $\vev{\mathcal{N}}$ can be expressed as
\begin{eqnarray}
\label{mean_nA}
 \bar{\mathcal{N}} = \bra{1} A \ket{w} = \frac{\bra{1} A \lr{1 - A}^{-1} \ket{b}}{\bra{1} \lr{1 - A}^{-1} \ket{b}} .
\end{eqnarray}
In practice, one can perform a linear mapping $\ket{\phi} = C \ket{\phi'}$ of variables in order to minimize $\bar{\mathcal{N}}$ and thus enhance the performance of the algorithm. In terms of the new variables, the original linear system of equations looks like $\ket{\phi'} = A' \ket{\phi'} + \ket{b'} = C^{-1} A C \ket{\phi'} + C^{-1} \ket{b}$. Let us now calculate $\bar{\mathcal{N}}$ for the new system:
\begin{eqnarray}
\label{mean_nA_new}
 \bar{\mathcal{N}}' = \bra{1} A' \ket{w'}
 =
 \frac{\bra{1} C^{-1} A \lr{1 - A}^{-1} \ket{b}}{\bra{1} C^{-1} \lr{1 - A}^{-1} \ket{b}}
 = \nonumber \\ =
 \frac{\bra{1} C^{-1} A \ket{\phi}}{\bra{1} C^{-1} \ket{\phi}}
 =
 \frac{\bra{1} C^{-1} \lr{\ket{\phi} - \ket{b}} }{\bra{1} C^{-1} \ket{\phi}}
 = \nonumber \\ =
 1 - \frac{\bra{1} C^{-1} \ket{b}}{\bra{1} C^{-1} \ket{\phi}}
 =
 1 - \frac{\bra{1} C^{-1} \ket{b}}{\bra{1} C^{-1} \ket{w}} \, \frac{1 - \bar{\mathcal{N}}}{\bra{1}\ket{b}}.
\end{eqnarray}

 Let us now apply the above general formulae to the prediction of $\bar{\mathcal{N}} \equiv \vev{nA}$ in the case of large-N Schwinger-Dyson equations. In this case we typically introduce the rescaled functions $w\lr{X}$ which are related to the field correlators $G\lr{X}$ as $G\lr{X} = \mathcal{N} c\lr{X} w\lr{X}$. Typically, the behaviour of the MC process quite strongly depends on the choice of $\mathcal{N}$ and $c\lr{X}$, and the question is how to predict $\vev{nA}$ for some values $c'\lr{X}$ and $\mathcal{N}'$. The functions $w'\lr{X}$ with these new values are related to $w\lr{X}$ as
\begin{eqnarray}
\label{wnew_wold}
 w'\lr{X} = \frac{\mathcal{N}}{\mathcal{N}'} \frac{c\lr{X}}{c'\lr{X}} w\lr{X} \equiv C^{-1} w\lr{X}
\end{eqnarray}

 In order to use the expression (\ref{mean_nA_new}), we have to find first
\begin{eqnarray}
\label{nA_prediction_denominator1}
 \bra{1} C^{-1} \ket{w} = \sum\limits_{m=1}^{+\infty} \lr{\frac{\mathcal{N}}{\mathcal{N}'}}^m
 \times \nonumber \\ \times
 \sum\limits_{X_m \ldots X_0}
 \frac{c\lr{X_m}}{c'\lr{X_m}} \ldots \frac{c\lr{X_0}}{c'\lr{X_0}} P\lr{X_m, \ldots, X_0} ,
\end{eqnarray}
where $P\lr{X_m, \ldots, X_0}$ is the probability to get the ``sequence of sequences'' $X_m, \ldots, X_0$:
\begin{eqnarray}
\label{overall_probability}
 P\lr{X_m, \ldots, X_0} = \mathcal{N}_p^{-1} w\lr{X_m} \ldots w\lr{X_0} ,
\end{eqnarray}
where
\begin{eqnarray}
\label{Np_def}
 \mathcal{N}_p = \frac{\mathcal{N}_b}{1 - \vev{nA}}, \quad
 \mathcal{N}_b = \sum\limits_X b\lr{X} .
\end{eqnarray}
Inserting (\ref{overall_probability}) into (\ref{nA_prediction_denominator}), we get
\begin{eqnarray}
\label{nA_prediction_denominator}
 \bra{1} C^{-1} \ket{w} = \frac{1}{\mathcal{N}_p} \, \frac{\frac{\mathcal{N}}{\mathcal{N}'} \eta}{1 - \frac{\mathcal{N}}{\mathcal{N}'} \eta},
\end{eqnarray}
where we define $\eta = \sum\limits_x \frac{c\lr{X}}{c'\lr{X}} w\lr{X}$. Inserting this result into (\ref{mean_nA_new}) and taking into account the definition (\ref{Np_def}), we get
\begin{eqnarray}
\label{nA_prediction1}
 \vev{nA}' = 1 - \frac{\mathcal{N}_b' \lr{\mathcal{N}' - \mathcal{N} \eta}}{\mathcal{N} \eta} .
\end{eqnarray}
In order to make this expression more explicit and suitable for real calculations, we can explicitly take into account the dependence of $\mathcal{N}_b'$ on $\mathcal{N}'$:
\begin{eqnarray}
\label{new_source_norm}
 \mathcal{N}_b' = \sum\limits_x C^{-1} b\lr{x} = \frac{\mathcal{N}}{\mathcal{N}'} \beta,
 \quad \beta = \sum\limits_x \frac{c\lr{X}}{c'\lr{X}} b\lr{X} ,
\end{eqnarray}
where now $\beta$ depends only on $c\lr{X}/c'\lr{X}$ but not on $\mathcal{N}/\mathcal{N}'$. Combining (\ref{nA_prediction1}) with (\ref{new_source_norm}), we get
\begin{eqnarray}
\label{nA_prediction}
 \vev{nA}' = 1 - \beta \, \lr{\frac{1}{\eta} - \frac{\mathcal{N}}{\mathcal{N}'}}
\end{eqnarray}

Also it is convenient to express $\eta$ in terms of probability distribution $\pi\lr{X}$ of the topmost element in the sequence:
\begin{eqnarray}
\label{topmost_distrib}
 w\lr{X} = \frac{\mathcal{N}_p}{1 + \mathcal{N}_p} \, \pi\lr{X} ,
\end{eqnarray}
from which we immediately get
\begin{eqnarray}
\label{topmost_distrib}
 \eta = \frac{\mathcal{N}_p}{1 + \mathcal{N}_p} \, \vev{ \frac{c\lr{X}}{c'\lr{X}} } ,
\end{eqnarray} .

\section{Schwinger-Dyson equations for $\phi^4$ matrix model beyond the planar approximation}
\label{sec:phi4sd_nonplanar}

\begin{eqnarray}
\label{phi4sd_nonplanar}
 G\lr{n_1, \ldots, n_m}
 = \nonumber \\ =
 2 G\lr{n_1 - 2, n_2, \ldots, n_m}
 + \nonumber \\ +
 \sum\limits_{a=1}^{n_1-2}
 G\lr{a, n_1 - 2 - a, n_2, \ldots, n_m}
 + \nonumber \\ +
 \lambda G\lr{n_1 + 2, n_2, \ldots, n_m}
 + \nonumber \\ +
 \frac{1}{N^2} \sum\limits_{A=2}^{m} n_A
 G\lr{n_2, \ldots, n_{A-1}, n_A + n_1 - 2, n_{A+1}, \ldots, n_m}
\end{eqnarray}

Basic operations:
\begin{description}
  \item[Create single trace:] $\lrc{n_1, \ldots, n_m} \rightarrow \lrc{2, n_1, n_2, \ldots, n_m}$
  \item[Create two traces:]
  \item[Insert line:] $\lrc{n_1, \ldots, n_m} \rightarrow \lrc{n_1 + 2, n_2, \ldots, n_m}$
  \item[Merge two traces:] $\lrc{n_1, \ldots, n_m} \rightarrow \lrc{n_1 + n_2 + 2, n_3, \ldots, n_m}$
  \item[Create vertex:] $\lrc{n_1, \ldots, n_m} \rightarrow \lrc{n_1 - 2, n_2, \ldots, n_m}$
  \item[Split single trace operator:] $\lrc{n_1, \ldots, n_m} \rightarrow \lrc{a, n_1, \ldots, n_{A-1}, n_A + 2 - a, n_{A+1}, \ldots, n_m}$
\end{description}

\begin{acknowledgments}
 This work was supported by the S.~Kowalevskaja award from the Alexander von Humboldt foundation.
\end{acknowledgments}

\bibliography{MyBibliography}
\bibliographystyle{apsrev}

\end{document}
