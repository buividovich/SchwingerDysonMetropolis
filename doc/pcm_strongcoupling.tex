\documentclass[12pt]{article}

\usepackage[a4paper,landscape,dvips]{geometry}

\usepackage{amsmath}
\usepackage{amsfonts}
\usepackage{color}

\newcommand{\comment}[1]{}

\newcommand{\lr}[1]{ \left( #1 \right) }
\newcommand{\lrs}[1]{ \left[ #1 \right] }
\newcommand{\lrc}[1]{ \left\{ #1 \right\} }
\newcommand{\vev}[1]{ \langle \, #1 \, \rangle }

\newcommand{\Tr}{ {\rm Tr} \, }
\newcommand{\tr}{ {\rm Tr} \, }
\newcommand{\re}{ {\rm Re} \, }
\newcommand{\im}{ {\rm Im} \, }
\renewcommand{\Re}{ {\rm Re} \, }
\renewcommand{\Im}{ {\rm Im} \, }

\newcommand{\rvac}{ \, | 0 \rangle }
\newcommand{\lvac}{ \langle 0 | \, }
\newcommand{\ket}[1]{ \, | #1 \rangle }
\newcommand{\bra}[1]{ \langle #1 | \, }

\newcommand{\diag}[1]{ {\rm diag} \, \left( #1 \right) }
\newcommand{\const}{ {\rm const}}
\renewcommand{\det}[1]{ {\rm det} \left( #1 \right) }

\newcommand{\sign}{ {\rm sign} \,  }
\newcommand{\sh}{ {\rm sh} \,  }
\newcommand{\ch}{ {\rm ch} \,  }
\renewcommand{\th}{ {\rm th} \,  }
\newcommand{\hodge}{{}^{*}}
\newcommand{\expa}[1]{ \exp{\left( #1 \right)} }
\newcommand{\abs}[1]{| #1 |}

\newcommand{\red}[1]{ \textcolor{red}{#1} }

\newcommand{\G}{\mathcal{G}}

\begin{document}

\section*{Generalized expansion from strong-coupling Schwinger-Dyson (no explicit action involved)}

\subsection*{Schwinger-Dyson equations}

Our starting point are the SD equations for the following correlators in the PCM model
\begin{eqnarray}
\label{gf_def}
\G\lr{x_0, y_0, \ldots, x_{n-1}, y_{n-1}}
=
\frac{1}{N}\, \vev{ \tr\lr{
g_{x_0} g^{\dag}_{y_0} \ldots g_{x_{n-1}} g^{\dag}_{y_{n-1}}} }
\end{eqnarray}
\begin{eqnarray}
\label{SDs_n2}
G\lr{x_0, y_0} =
\delta_{x_0, y_0}
 - %\nonumber \\ -
\frac{1}{\lambda} D_{x_0 x} \, \G\lr{x, y_0}
 +
\frac{1}{\lambda} D_{x_0 x} \, \G\lr{x_0, x, x_0, y_0}
\end{eqnarray}

\begin{eqnarray}
\label{SDs}
\G\lr{x_0, y_0, \ldots, x_{n-1}, y_{n-1}}
= %\nonumber\\ =
\sum\limits_{A=1}^{n-2} \delta_{x_0, y_A} \,
\G\lr{x_A, y_0, \ldots, x_{A-1}, y_{A-1}}\,
\G\lr{x_{A+1}, y_{A+1}, \ldots, x_{n-1}, y_{n-1}}\,
+ \nonumber \\ +
\delta_{x_0, y_0} \, \G\lr{x_1, y_1, \ldots, x_{n-1}, y_{n-1}}\,
+
\delta_{x_0, y_{n-1}} \, \G\lr{x_{n-1}, y_0, \ldots, x_{n-2}, y_{n-2}}\,
- \nonumber \\ -
\sum\limits_{A=1}^{n-1} \delta_{x_0, x_A} \,
 \G\lr{x_0, y_0, \ldots, x_{A-1}, y_{A-1}}\,
 \G\lr{x_A, y_A, \ldots, x_{n-1}, y_{n-1}}\,
- \nonumber \\ -
\frac{1}{\lambda} D_{x_0 x} \G\lr{x, y_0, \ldots, x_n, y_n}
+
\frac{1}{\lambda} D_{x_0 y} \, \G\lr{x_0, y, x_0, y_0, \ldots, x_{n-1}, y_{n-1}}
\end{eqnarray}

We can also represent these equations in the following form:
\begin{eqnarray}
\label{SDs_n2_1}
\sum\limits_x \lr{\delta_{\bar{x}_0 x} + \frac{1-\alpha}{\lambda} D_{\bar{x}_0 x} } G\lr{x, y_0} =
\delta_{\bar{x}_0, y_0}
 - %\nonumber \\ -
\frac{\alpha}{\lambda} D_{\bar{x}_0 x} \, \G\lr{x, y_0}
 +
\frac{1}{\lambda} D_{\bar{x}_0 x} \, \G\lr{\bar{x}_0, x, \bar{x}_0, y_0}
\end{eqnarray}

\begin{eqnarray}
\label{SDs_1}
\sum\limits_x \lr{\delta_{\bar{x}_0 x} + \frac{1-\alpha}{\lambda} D_{\bar{x}_0 x} } \G\lr{x, y_0, \ldots, x_{n-1}, y_{n-1}}
= %\nonumber\\ =
\sum\limits_{A=1}^{n-2} \delta_{\bar{x}_0, y_A} \,
\G\lr{x_A, y_0, \ldots, x_{A-1}, y_{A-1}}\,
\G\lr{x_{A+1}, y_{A+1}, \ldots, x_{n-1}, y_{n-1}}\,
+ \nonumber \\ +
\delta_{\bar{x}_0, y_0} \, \G\lr{x_1, y_1, \ldots, x_{n-1}, y_{n-1}}\,
+
\delta_{\bar{x}_0, y_{n-1}} \, \G\lr{x_{n-1}, y_0, \ldots, x_{n-2}, y_{n-2}}\,
- \nonumber \\ -
\sum\limits_{A=1}^{n-1} \delta_{\bar{x}_0, x_A} \,
 \G\lr{\bar{x}_0, y_0, \ldots, x_{A-1}, y_{A-1}}\,
 \G\lr{x_A, y_A, \ldots, x_{n-1}, y_{n-1}}\,
- \nonumber \\ -
\frac{\alpha}{\lambda} D_{\bar{x}_0 x} \G\lr{x, y_0, \ldots, x_n, y_n}
+
\frac{1}{\lambda} D_{\bar{x}_0 x} \, \G\lr{\bar{x}_0, x, \bar{x}_0, y_0, \ldots, x_{n-1}, y_{n-1}}
\end{eqnarray}

 Introducing now the ``propagator'' $G^0_{xy}$ via the identity
\begin{eqnarray}
\label{propagator_coord_def}
 \sum\limits_x \lr{\delta_{x z} + \frac{1-\alpha}{\lambda} D_{x z} } G^0_{z y} = \delta_{x y}
\end{eqnarray}
and summing the equations (\ref{SDs_1}) and (\ref{SDs_n2_1}) over $\bar{x}_0$ with the weight $G^0_{x_0 \bar{x}_0}$, we get
\begin{eqnarray}
\label{SDs_n2_1}
G\lr{x_0, y_0} =
G^0_{x_0, y_0}
 - %\nonumber \\ -
\frac{\alpha}{\lambda} \lrs{G^0 D}_{x_0 x} \, \G\lr{x, y_0}
 +
\frac{1}{\lambda} \sum\limits_{\bar{x}_0 x} G^0_{x_0 \bar{x}_0} D_{\bar{x}_0 x} \, \G\lr{\bar{x}_0, x, \bar{x}_0, y_0}
\end{eqnarray}

\begin{eqnarray}
\label{SDs_1}
\G\lr{x_0, y_0, \ldots, x_{n-1}, y_{n-1}}
= %\nonumber\\ =
\sum\limits_{A=1}^{n-2} G^0_{x_0, y_A} \,
\G\lr{x_A, y_0, \ldots, x_{A-1}, y_{A-1}}\,
\G\lr{x_{A+1}, y_{A+1}, \ldots, x_{n-1}, y_{n-1}}\,
+ \nonumber \\ +
G^0_{x_0, y_0} \, \G\lr{x_1, y_1, \ldots, x_{n-1}, y_{n-1}}\,
+
G^0_{x_0, y_{n-1}} \, \G\lr{x_{n-1}, y_0, \ldots, x_{n-2}, y_{n-2}}\,
- \nonumber \\ -
\sum\limits_{A=1}^{n-1} G^0_{x_0, x_A} \,
 \G\lr{x_A, y_0, \ldots, x_{A-1}, y_{A-1}}\,
 \G\lr{x_A, y_A, \ldots, x_{n-1}, y_{n-1}}\,
- \nonumber \\ -
\frac{\alpha}{\lambda} \lrs{G D}_{x_0 x} \G\lr{x, y_0, \ldots, x_n, y_n}
+
\frac{1}{\lambda} \sum\limits_{\bar{x}_0, x} G^0_{x_0 \bar{x}_0} D_{\bar{x}_0 x} \, \G\lr{\bar{x}_0, x, \bar{x}_0, y_0, \ldots, x_{n-1}, y_{n-1}}
\end{eqnarray}


\subsection*{Weak-coupling-like expansion}

Let us now define the correlators in the momentum space:
\begin{eqnarray}
\label{momentum_space_def}
 G\lr{p_1, q_1, \ldots, p_n, q_n} = \frac{\lambda^{-n}}{V^{2 n}} \,
\sum\limits_{x_1, y_1} \ldots \sum\limits_{x_n, y_n} \,
\expa{i \sum\limits_A p_A x_A + i \sum\limits_A q_A y_A} \,
\G\lr{x_1, y_1, \ldots, x_n, y_n}  ,
\end{eqnarray}
or, conversely,
\begin{eqnarray}
\label{momentum_space_def_inverse}
 \G\lr{x_1, y_1, \ldots, x_n, y_n}
 =
 \lambda^{n}
 \sum\limits_{p_1, q_1} \ldots \sum\limits_{p_n, q_n} \,
 \expa{-i \sum\limits_A p_A x_A - i \sum\limits_A q_A y_A}
  G\lr{p_1, q_1, \ldots, p_n, q_n} .
\end{eqnarray}

In the momentum space the equations (\ref{SDs_n2}), (\ref{SDs}) read:
\begin{eqnarray}
\label{SDs_n2_momentum}
 G\lr{p_1, q_1} = \frac{\G_0\lr{p_1} \delta\lr{p_1 + q_1}}{V}
 +
 \lambda G_0\lr{p_1} \,
\sum\limits_{\tilde{p}_1, \tilde{q}_1, \tilde{p}_2}
\delta\lr{p_1, \tilde{p}_1 + \tilde{q}_1 + \tilde{p}_2} \,
D\lr{\tilde{q}_1} G\lr{\tilde{p}_1, \tilde{q}_1, \tilde{p}_2, q_1}
\end{eqnarray}

\begin{eqnarray}
\label{SDs_pcm_momentum}
 G\lr{p_1, q_1, \ldots, p_n, q_n}
 = \nonumber \\ =
 \sum\limits_{A=2}^{n-1}
 \frac{G_0\lr{p_1} \, \delta\lr{p_1 + q_A}}{V} \,
 G\lr{    p_A,     q_1, \ldots, p_{A-1}, q_{A-1}}
 G\lr{p_{A+1}, q_{A+1}, \ldots,     p_n,    q_n }
 + \nonumber \\ +
 \frac{G_0\lr{p_1} \, \delta\lr{p_1 + q_1}}{V} \,
 G\lr{p_2, q_2, \ldots, p_n, q_n}
 + \nonumber \\ +
 \frac{G_0\lr{p_1} \, \delta\lr{p_1 + q_n}}{V} \,
 G\lr{p_n, q_1, p_2, q_2, \ldots, p_{n-1}, q_{n-1}}
 - \nonumber \\ -
 \lambda \, \frac{G_0\lr{p_1}}{V} \,
 \sum\limits_{A=2}^{n}
 \sum\limits_{\tilde{p}_1 \tilde{p}_A} \delta\lr{p_1 + p_A, \tilde{p}_1 + \tilde{p}_A}
 G\lr{\tilde{p}_1, q_1, p_2, q_2, \ldots, p_{A-1}, q_{A-1}}
 G\lr{\tilde{p}_A, q_A,           \ldots, p_n, q_n}
 + \nonumber \\ +
 \lambda G_0\lr{p_1}
 \sum\limits_{\tilde{p}_1, \tilde{q}_1, \tilde{p_2}}
 \delta\lr{p_1, \tilde{p}_1 + \tilde{q}_1 + \tilde{p}_2} D\lr{\tilde{q}_1}
 G\lr{\tilde{p}_1, \tilde{q}_1, \tilde{p}_2, q_1, p_2, q_2, \ldots, p_n, q_n}
\end{eqnarray}
where we have defined the effective propagator $\G_0\lr{p} = \lr{\lambda + D\lr{p}}^{-1}$.

Exploring now the freedom of shifting $D\lr{p} \rightarrow D\lr{p} + c$, which is ensured by the unitarity of the $g$ matrices, let us now define $D\lr{p} = D_0\lr{p} - \lambda + m_0^2$, where $D_0\lr{p}$ behaves as $p^2$ at small $p$. As a result, the bare propagator reads $\G_0\lr{p} = \lr{m_0^2 + D_0\lr{p}}^{-1}$.

\subsection*{Leading order (mean-field) result}

For the leading-order result for $\G_{x y}$ we trivially find
\begin{eqnarray}
\label{order0}
 G_{x y} = \lambda \int \frac{d^2 q}{\lr{2 \pi}^2} \frac{e^{i q \lr{x - y}}}{m_0^2 + D_0\lr{q}} .
\end{eqnarray}
Tuning now the mass in order to achieve $G_{xx} = 1$, we get the standard mean-field equation for the bare mass $m_0$, $\lambda I_0\lr{m_0} = 1$, where $I_0 = \int \frac{d^d q}{\lr{2 \pi}^d} G_0\lr{q}$. In 1D and 2D we have, correspondingly
\begin{eqnarray}
\label{lattice_I0_1D}
 I_0\lr{m} = \frac{1}{m \sqrt{4 + m^2}} \quad (d = 1)
 \nonumber \\
\label{lattice_I0_2D}
 I_0\lr{m} = \frac{2 \text{EllipticK}\left[-\frac{16}{8 m^2 + m^4}\right]}{\pi  \sqrt{8 m^2 + m^4}} =
 \frac{1}{4 \pi} \ln\lr{\frac{32}{m^2}} + O\lr{m^2 \ln\lr{m}}
\end{eqnarray}

\subsection*{Next-to-leading order result}

 For the next correction it is easy to find
\begin{eqnarray}
\label{nlo_Gpq}
 G^{\lr{1}}\lr{p_1 q_1} = \lambda \, \frac{G_0^2\lr{p_1} \delta\lr{p_1 + q_1}}{V}
 \, 2 \, \frac{1}{V} \sum\limits_q D\lr{q} G_0\lr{q} .
\end{eqnarray}
Taking into account the exact form of $D\lr{q}$, we find for the NLO propagator
\begin{eqnarray}
\label{nlo_Gpq}
 G_1\lr{p} = G_0\lr{p} + \lambda G_0^2\lr{p} \, 2 \, \lr{1 - \lambda I_0}
\end{eqnarray}
Calculating now again $\G_{xx}$ we get
\begin{eqnarray}
\label{nlo_Gxx}
 G_{xx} = \lambda \lr{I_0 + 2 \lambda I_1 \lr{1 - \lambda I_0}} ,
\end{eqnarray}
where we have defined $I_1\lr{m} = \int \frac{d^d q}{\lr{2 \pi}^d} G_0\lr{q}^2$:
\begin{eqnarray}
\label{lattice_I1_1D}
  I_1\lr{m} = \frac{2 + m^2}{m^3 \lr{4 + m^2}^{3/2}} \quad (d = 1)
 \nonumber \\
\label{lattice_I0_2D}
 I_1\lr{m} = \frac{1}{4 \pi m^2} + O\lr{m^2} \quad (d = 2) .
\end{eqnarray}
An obvious solution to $G_{xx} = 1$ is again $\lambda I_0 = 1$, and hence $G_1\lr{p}$ does not receive corrections at this order! This can be easily understood if one remembers that the leading-order result here is the same as for the $O\lr{N}$ sigma model, where the answer is given by cactus diagrams and mean-field is known to be exact. The leading order correction here is the tadpole - and it is a cactus-type diagram!

 Let us also consider the expansion of the mean link:
\begin{eqnarray}
\label{mean_link_lo}
 \vev{g^{\dag}_0 g_1} = 1 - \frac{\lambda}{4} + \frac{m^2}{4}
\end{eqnarray}

\subsection*{Next-to-next-to-leading-order result}

 For the next-order result we have to calculate $G_1\lr{p_1 q_1 p_2 q_2}$. The last summand takes the form
\begin{eqnarray}
\label{G4_nlo1}
 \sum\limits_{\tilde{p}_1, \tilde{q}_1} D\lr{\tilde{q}_1} G_0\lr{\tilde{p}_1} G_0\lr{p_1 - \tilde{p}_1 - \tilde{q}_1} G_0\lr{p_2}
 \delta\lr{\tilde{p}_1 + q_1} \delta\lr{p_1 - \tilde{p}_1 - \tilde{q}_1 + \tilde{q}_1} \delta\lr{p_2 + q_2}
 + \nonumber \\ +
 \sum\limits_{\tilde{p}_1, \tilde{q}_1} D\lr{\tilde{q}_1} G_0\lr{\tilde{p}_1} G_0\lr{p_1 - \tilde{p}_1 - \tilde{q}_1} G_0\lr{p_2}
 \delta\lr{\tilde{p}_1 + \tilde{q}_1} \delta\lr{p_1 - \tilde{p}_1 - \tilde{q}_1 + q_1} \delta\lr{p_2 + q_2}
 + \nonumber \\ +
 \sum\limits_{\tilde{p}_1, \tilde{q}_1} D\lr{\tilde{q}_1} G_0\lr{\tilde{p}_1} G_0\lr{p_1 - \tilde{p}_1 - \tilde{q}_1} G_0\lr{p_2}
 \delta\lr{\tilde{p}_1 + \tilde{q}_1} \delta\lr{p_1 - \tilde{p}_1 - \tilde{q}_1 + q_2} \delta\lr{p_2 + q_1}
 + \nonumber \\ +
 \sum\limits_{\tilde{p}_1, \tilde{q}_1} D\lr{\tilde{q}_1} G_0\lr{\tilde{p}_1} G_0\lr{p_1 - \tilde{p}_1 - \tilde{q}_1} G_0\lr{p_2}
 \delta\lr{\tilde{p}_1 + q_2} \delta\lr{p_1 - \tilde{p}_1 - \tilde{q}_1 + q_1} \delta\lr{p_2 + \tilde{q}_1}
 + \nonumber \\ +
 \sum\limits_{\tilde{p}_1, \tilde{q}_1} D\lr{\tilde{q}_1} G_0\lr{\tilde{p}_1} G_0\lr{p_1 - \tilde{p}_1 - \tilde{q}_1} G_0\lr{p_2}
 \delta\lr{\tilde{p}_1 + q_2} \delta\lr{p_1 - \tilde{p}_1 - \tilde{q}_1 + \tilde{q}_1} \delta\lr{p_2 + q_1} ,
\end{eqnarray}
which simplifies to (after multiplication by $G_0\lr{p_1}$)
\begin{eqnarray}
\label{G4_nlo2}
 2 \frac{G_0^2\lr{p_1} \delta\lr{p_1 + q_2}}{V} \frac{G_0\lr{p_2} \delta\lr{p_2 + q_1}}{V} \lr{1 - \lambda I_0} +
 2 \frac{G_0^2\lr{p_1} \delta\lr{p_1 + q_1}}{V} \frac{G_0\lr{p_2} \delta\lr{p_2 + q_2}}{V} \lr{1 - \lambda I_0}
 + \nonumber \\ +
 D\lr{p_2} G_0\lr{p_1} G_0\lr{q_1} G_0\lr{p_2} G_0\lr{q_2} \delta\lr{p_1 + q_1 + p_2 + q_2}
\end{eqnarray}
Here we use the identity $D\lr{p} G_0\lr{p} = 1 - \lambda G_0\lr{p}$. The summand corresponding to $1$ then cancels with the double-trace interaction vertex in this formula, and we are left with the standard four-momentum interaction vertex plus the disconnected tadpole contributions:
\begin{eqnarray}
\label{G4_nlo1}
 G^{\lr{1}}\lr{p_1 q_1 p_2 q_2} =
 \frac{\delta\lr{p_1 + q_1}}{V} \, \frac{\delta\lr{p_2 + q_2}}{V}
 \lr{G_0\lr{p_1} G^{\lr{1}}\lr{p_2} + G^{\lr{1}}\lr{p_1} G_0\lr{p_2} }
 + \nonumber \\ +
 \frac{\delta\lr{p_1 + q_2}}{V} \, \frac{\delta\lr{p_2 + q_1}}{V}
 \lr{G_0\lr{p_1} G^{\lr{1}}\lr{p_2} + G^{\lr{1}}\lr{p_1} G_0\lr{p_2} }
 - \nonumber \\ -
 \lambda^2 G_0\lr{p_1} G_0\lr{p_2} G_0\lr{q_1} G_0\lr{q_2} \frac{\delta\lr{p_1 + q_1 + p_2 + q_2}}{V^3}
\end{eqnarray}
We can now calculate the next-to-next-to-leading order contribution to $G\lr{p_1 q_1}$. In a compact form we can write
\begin{eqnarray}
\label{Gp_nnlo1}
 G^{\lr{2}}\lr{p} = G_0^3\lr{p} \Sigma_1^2 + G_0^2\lr{p} \Sigma_2
 - \lambda G_0^2\lr{p} V^{-2}\sum\limits_{\tilde{p}, \tilde{q}} G_0\lr{\tilde{p}} D\lr{\tilde{q}} G_0\lr{\tilde{q}} G_0\lr{p - \tilde{p} - \tilde{q}} ,
\end{eqnarray}
where we have denoted
\begin{eqnarray}
\label{sigmas_def}
 \Sigma_1 = 2 \vev{D\lr{p} G_0\lr{p}} = 2 \lr{1 - \lambda I_0},
 \nonumber \\
 \Sigma_2 = 2 \vev{D\lr{p} G_0^2\lr{p}} \Sigma_1 = 2 \lr{I_0 - \lambda I_1} \Sigma_1 , \quad I_1 \equiv \vev{G_0^2\lr{p}}
\end{eqnarray}
Transforming the term $D\lr{\tilde{q}} G_0\lr{\tilde{q}}$, we arrive at
\begin{eqnarray}
\label{Gp_nnlo1}
 G^{\lr{2}}\lr{p} = G_0^3\lr{p} \Sigma_1^2 + G_0^2\lr{p} \Sigma_2
 - \lambda I_0^2 G_0^2\lr{p}
  +
  \lambda^2 G_0^2\lr{p} V^{-2}\sum\limits_{\tilde{p}, \tilde{q}} G_0\lr{\tilde{p}} G_0\lr{\tilde{q}} G_0\lr{p - \tilde{p} - \tilde{q}}
\end{eqnarray}
For $G_{xx}$ we now get
\begin{eqnarray}
\label{Gp_nnlo1}
 G_{xx}^{\lr{2}} = I_2 \Sigma_1^2 + I_1 \Sigma_2
 - \lambda I_0^2 I_1
  +
  \lambda^2 \vev{ G_0^2\lr{p} G_0\lr{\tilde{p}} G_0\lr{\tilde{q}} G_0\lr{p - \tilde{p} - \tilde{q}} }
\end{eqnarray}

\subsection*{Strong-coupling-like expansion}

 To arrive at the conventional strong-coupling-like expansion, we use somewhat different conventions for correlators in momentum space:
\begin{eqnarray}
\label{momentum_space_def}
 G\lr{p_0, q_0, \ldots, p_{n-1}, q_{n-1}} = \frac{1}{V^{2 n}} \,
\sum\limits_{x_0, y_0} \ldots \sum\limits_{x_{n-1}, y_{n-1}} \,
\expa{i \sum\limits_A p_A x_A + i \sum\limits_A q_A y_A} \,
\G\lr{x_0, y_0, \ldots, x_{n-1}, y_{n-1}}  ,
\end{eqnarray}
or, conversely,
\begin{eqnarray}
\label{momentum_space_def_inverse}
 \G\lr{x_0, y_0, \ldots, x_{n-1}, y_{n-1}}
 =
 \sum\limits_{p_0, q_0} \ldots \sum\limits_{p_{n-1}, q_{n-1}} \,
 \expa{-i \sum\limits_A p_A x_A - i \sum\limits_A q_A y_A}
  G\lr{p_0, q_0, \ldots, p_{n-1}, q_{n-1}} .
\end{eqnarray}

Correspondingly, the Schwinger-Dyson equations in momentum space take the following form:
\begin{eqnarray}
\label{SDs_n2_momentum}
 G\lr{p_0, q_0} = \frac{\G_0\lr{p_0} \delta\lr{p_0 + q_0}}{V}
 +
 \red{\lambda^{-1}} \, G_0\lr{p_0} \,
\sum\limits_{\tilde{p}_0, \tilde{q}_0, \tilde{p}_1}
\delta\lr{p_0, \tilde{p}_0 + \tilde{q}_0 + \tilde{p}_1} \,
D\lr{\tilde{q}_0} G\lr{\tilde{p}_0, \tilde{q}_0, \tilde{p}_1, q_0}
\end{eqnarray}

\begin{eqnarray}
\label{SDs_pcm_momentum}
 G\lr{p_0, q_0, \ldots, p_{n-1}, q_{n-1}}
 = \nonumber \\ =
 \sum\limits_{A=1}^{n-2}
 \frac{G_0\lr{p_0} \, \delta\lr{p_0 + q_A}}{V} \,
 G\lr{    p_A,     q_0, \ldots, p_{A-1}, q_{A-1}}
 G\lr{p_{A+1}, q_{A+1}, \ldots,     p_{n-1},    q_{n-1} }
 + \nonumber \\ +
 \frac{G_0\lr{p_0} \, \delta\lr{p_0 + q_0}}{V} \,
 G\lr{p_1, q_1, \ldots, p_{n-1}, q_{n-1}}
 + \nonumber \\ +
 \frac{G_0\lr{p_0} \, \delta\lr{p_0 + q_{n-1}}}{V} \,
 G\lr{p_{n-1}, q_0, p_1, q_1, \ldots, p_{n-2}, q_{n-2}}
 - \nonumber \\ -
 \frac{G_0\lr{p_0}}{V} \,
 \sum\limits_{A=1}^{n-1}
 \sum\limits_{\tilde{p}_0 \tilde{p}_A} \delta\lr{p_0 + p_A, \tilde{p}_0 + \tilde{p}_A}
 G\lr{\tilde{p}_0, q_0, p_1, q_1, \ldots, p_{A-1}, q_{A-1}}
 G\lr{\tilde{p}_A, q_A,           \ldots, p_{n-1}, q_{n-1}}
 + \nonumber \\ +
 \red{\lambda^{-1}} \, G_0\lr{p_0}
 \sum\limits_{\tilde{p}_0, \tilde{q}_0, \tilde{p_1}}
 \delta\lr{p_0, \tilde{p}_0 + \tilde{q}_0 + \tilde{p}_1} D\lr{\tilde{q}_0}
 G\lr{\tilde{p}_0, \tilde{q}_0, \tilde{p}_1, q_0, p_1, q_1, \ldots, p_{n-1}, q_{n-1}}
\end{eqnarray}
where we have defined the effective propagator $\G_0\lr{p} = \lr{1 + \red{\lambda^{-1}} D\lr{p}}^{-1}$.

\end{document}

